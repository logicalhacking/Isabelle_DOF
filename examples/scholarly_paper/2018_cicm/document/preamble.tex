% This file was modified by the DOF LaTeX converter, version 0.0.3
\usepackage[scaled=0.88]{beramono}
\usepackage{upquote}
\usepackage{textcomp}
\usepackage{xcolor}
\usepackage{paralist}
\usepackage{listings}
\usepackage{lstisadof}
\usepackage{xspace}
\usepackage[draft]{fixme}

\lstloadlanguages{bash}
\lstdefinestyle{bash}{language=bash,
  ,basicstyle=\ttfamily%
  ,showspaces=false%
  ,showlines=false%
  ,columns=flexible%
 %  ,keywordstyle=\bfseries%
   % Defining 2-keywords
   ,keywordstyle=[1]{\color{BrickRed!60}\bfseries}%
   % Defining 3-keywords
   ,keywordstyle=[2]{\color{OliveGreen!60}\bfseries}%
   % Defining 4-keywords
   ,keywordstyle=[3]{\color{black!60}\bfseries}%
   % Defining 5-keywords
   ,keywordstyle=[4]{\color{Blue!70}\bfseries}%
   % Defining 6-keywords
   ,keywordstyle=[5]{\itshape}%
  % 
}
\lstdefinestyle{displaybash}{style=bash,
                basicstyle=\ttfamily\footnotesize,
                backgroundcolor=\color{black!2}, frame=lines}%

\lstnewenvironment{bash}[1][]{\lstset{style=displaybash, #1}}{}
\def\inlinebash{\lstinline[style=bash, breaklines=true,columns=fullflexible]}

\usepackage[caption]{subfig}
\usepackage[size=footnotesize]{caption}

\newcommand{\ie}{i.e.}
\newcommand{\eg}{e.g.}


\subject{Example of an Academic Paper\footnote{%
  This document is an example setup for writing academic paper. While
  it is optimized for Springer's LNCS class, it uses a Koma Script
  LaTeX class to avoid the need for distributing \texttt{llncs.cls},
  which would violate Springer's copyright. This example has been
  published at CICM 2018:
    \protect\begin{quote}
      Achim D. Brucker, Idir Ait-Sadoune, Paolo Crisafulli, and
      Burkhart Wolff. Using The Isabelle Ontology Framework: Linking
      the Formal with the Informal. In Conference on Intelligent
      Computer Mathematics (CICM). Lecture Notes in Computer Science
      (11006), Springer-Verlag, 2018. 
    \protect\end{quote}
    Note that the content of this example is not updated and, hence,
    might not be correct with respect to the latest version of
    \isadof{}. 
    }}

%%% Local Variables:
%%% mode: latex
%%% TeX-master: "root.tex"
%%% End:
