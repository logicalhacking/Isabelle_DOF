\documentclass[11pt,a4paper]{book}
\usepackage{isabelle,isabellesym}
\usepackage{graphicx}
\graphicspath {{figures/}}

% further packages required for unusual symbols (see also
% isabellesym.sty), use only when needed


\usepackage{latexsym}
\usepackage{amssymb}
  %for \<leadsto>, \<box>, \<diamond>, \<sqsupset>, \<mho>, \<Join>,
  %\<lhd>, \<lesssim>, \<greatersim>, \<lessapprox>, \<greaterapprox>,
  %\<triangleq>, \<yen>, \<lozenge>

%\usepackage[greek,english]{babel}
  %option greek for \<euro>
  %option english (default language) for \<guillemotleft>, \<guillemotright>

%\usepackage[latin1]{inputenc}
  %for \<onesuperior>, \<onequarter>, \<twosuperior>, \<onehalf>,
  %\<threesuperior>, \<threequarters>, \<degree>

%\usepackage[only,bigsqcap]{stmaryrd}
  %for \<Sqinter>

%\usepackage{eufrak}
  %for \<AA> ... \<ZZ>, \<aa> ... \<zz> (also included in amssymb)

%\usepackage{textcomp}
  %for \<cent>, \<currency>

% this should be the last package used
\usepackage{pdfsetup}

% urls in roman style, theory text in math-similar italics
\urlstyle{rm}
\isabellestyle{it}
\newcommand{\HOL}[1]{\verb{HOL}}
\newcommand{\eg}[1]{e.g.}
\renewcommand{\isasymdegree}{XXX}


\begin{document}

\title{International Systems of Quantities and Units in Isabelle/HOL}
\author{Simon Foster \and Burkhart Wolff}
\maketitle

\textbf{ Abstract } 
The International System of Units 
(SI, abbreviated from the French Syst\`eme international (d’unit\'es)) is the modern form of 
the metric system and is the most widely used system of measurement. It comprises a coherent 
system of units of measurement built on seven base units, which are the second, metre, kilogram, 
ampere, kelvin, mole, candela, and a set of twenty prefixes to the unit names and unit symbols 
that may be used when specifying multiples and fractions of the units. The system also specifies 
names for 22 derived units, such as lumen and watt, for other common physical quantities. 

This theory represents a formal model of SI Units together with a deep integration in Isabelle's
type system: unitswere represented in a way that they have a \emph{unit type} comprising its 
\emph{magnitude type} and its physical \emph{dimension}. Congruences on dimensions were supported. 
Our construction is validated by a test-set of known equivalences between SI Units.
Moreover, the presented theory can be used for type-safe conversions between the SI system and
others, like the British Imperial System (BIS).

\tableofcontents

% sane default for proof documents
\parindent 0pt\parskip 0.5ex


\chapter{SI Units in Isabelle \\ An Introduction}

The International System of Units (SI, abbreviated from the French
Système international (d'unités)) is the modern form of the metric
system and is the most widely used system of measurement. It comprises
a coherent system of units of measurement built on seven base units,
which are the second, metre, kilogram, ampere, kelvin, mole, candela,
and a set of twenty prefixes to the unit names and unit symbols that
may be used when specifying multiples and fractions of the units.
The system also specifies names for 22 derived units, such as lumen and
watt, for other common physical quantities.

(cited from \url{https://en.wikipedia.org/wiki/International_System_of_Units}).

In more detail, the SI provides the following fundamental concepts:

%
\begin{enumerate}%
\item \emph{quantities}, i.e. \emph{time}, \emph{length}, \emph{mass}, \emph{electric current},
\emph{temperature}, \emph{amount of substance},\emph{luminous intensity},
and other derived quantities such as \emph{volume};

\item \emph{dimensions}, i.e. a set of the symbols  \isa{T}, \isa{L}, \isa{M}, \isa{I},  \isa{{\isasymTheta}}, \isa{N}, \isa{J}  corresponding
to the above mentioned base quantities,  indexed by an integer exponent
(dimensions were also called \emph{base unit names} or just \emph{base units});

\item \emph{magnitudes}, i.e. a factor or \emph{prefix}
(typically integers, reals, vectors on real or complex numbers);

\item \emph{units}, which are basically pairs of magnitudes and dimensions denoting quantities.
\end{enumerate}

Note that \emph{quantities} are understood as idealized  \emph{physical concepts}, and are sharply 
distinguished from  \emph{units} representing \emph{a means to measure} them. The difference is 
roughly similar between words (or: symbols) and notions. This distinction reflects the fact that
these concepts have been at times heavily questioned in the history of physics; the interested
reader is referred to the debate around the "relativistic mass" at in the twenties of the last century.

The purpose of the presented theory is to model SI units with polymorphic magnitudes inside the
Isabelle/HOL\cite{nipkow.ea:isabelle:2002} system. The objective of our construction is to
reflect the types of the magnitudes as well as their dimensions in order to allow type-safe 
calculations on SI units.

As a result, it is possible to express "4500.0 kilogram times meter per second square" which can
have the type 
\isa{{\isasymreal}\ {\isacharbrackleft}M\ \isactrlsup {\isachardot}\ L\ \isactrlsup 
 {\isachardot}\ T\isactrlsup {\isacharminus}\isactrlsup {\isadigit{3}} 
 \isactrlsup {\isachardot}\   T\isactrlsup {\isadigit{1}}{\isacharbrackright}}. 
For units of this type we can infer that this corresponds to the derived unit "4.5 kN" (kilo-Newton) 
 of type \isa{{\isasymreal}\ {\isacharbrackleft}M\ \isactrlsup {\isachardot}\ L\ \isactrlsup 
 {\isachardot}\ T\isactrlsup {\isacharminus}\isactrlsup {\isadigit{2}}{\isacharbrackright}}.  

This is an attempt to model the standard SI system and its derived entities (cf.
\url{https://www.quora.com/What-are-examples-of-SI-units}),
both from a type-checking as well as a proof-support perspective.
These design objectives are for the case of Isabelle system somewhat contradictory.
Since the Isabelle type system follows the Curry-style paradigm, which can be 
characterized by: "be as implicit as possible, inference as automatic as possible", it is not
possible to do computations on type-terms (in contrast to, for example, Coq). We therefore
need a more involved construction using Isabelle's type-classes to establish a semantic 
interpretation on certain classes of types. This paves the way to derive rules that
establish and exploit type isomorphisms implicitely. For example, the implicit
type isomorhisms for the dimensions $T^{-2} * T$ and $T^{-1}$ is dealt with appropriate
rules on terms and a special form of equivalence. 

Our construction proceeds in three phases:
\begin{enumerate}%
\item We construct a type \isa{Dimension} which is basically a "semantic representation" or
"semantic domain" of all SI dimensions. Since SI-types have an interpretation in this domain,
it serves to give semantics to type-constructors by operations on this domain, too.
We construct a multiplicative group on it.

\item From \isa{Dimension} we build a language of type-constructors of \isa{dimS}-types, 
 captured in a type class \isa{dim-types} giving it a pseudo-inductive structure. Types of
 this class are required to have  an interpretation function into \isa{Dimension}'s, which
 allows for establishing equivalences on \isa{dim-types}.
      
\item We construct a SI Unit-type as a pair of a polymorphic magnitude and a dimension type.
  This type will be inhabited by all basic SI units and predicates expressing their relationship.
\end{enumerate}%

On this basis, an algebra of SI units can be derived; while types can not be "touched" inside
an Isabelle/HOL logic, it is possible to transform unit types by applying rules of this algebra.
Tactical support over this algebra is provided enabling the construction of normal forms on 
both units and their type.  

\subsubsection{Previous Attempts.} The work of \cite{HayesBrendan95} represents to our knowledge a
first attempt to formalize SI units in Z, thus a similar language of HOL. While our typing
representation is more rigourous due to the use of type-classes, this works lacks any attempt
to support formal and automated deduction on Si unit equivalences.

MORE TO COME.

%
\begin{isabellebody}%
\setisabellecontext{paper}%
%
\isadelimtheory
%
\endisadelimtheory
%
\isatagtheory
%
\endisatagtheory
{\isafoldtheory}%
%
\isadelimtheory
%
\endisadelimtheory
%
\isadelimML
%
\endisadelimML
%
\isatagML
%
\endisatagML
{\isafoldML}%
%
\isadelimML
%
\endisadelimML

\begin{isamarkuptitle*}
[label = {tit},type = {scholarly_paper.title}, args={label = {tit},type = {scholarly_paper.title}, scholarly_paper.title.short_title = {}}]A Framework for Proving Ontology-Relations and Runtime Testing Ontology Instances
\end{isamarkuptitle*}\isamarkuptrue%

\begin{isamarkuptext*}
[label = {idir},type = {scholarly_paper.author}, args={label = {idir},type = {scholarly_paper.author}, scholarly_paper.author.email = {idir.aitsadoune@centralesupelec.fr}, scholarly_paper.author.orcid = {0000-0002-6484-8276}, scholarly_paper.author.affiliation = {Université Paris-Saclay, CentraleSupélec, LMF, France}, scholarly_paper.author.http_site = {}}]Idir Ait-Sadoune
\end{isamarkuptext*}\isamarkuptrue%

\begin{isamarkuptext*}
[label = {nic},type = {scholarly_paper.author}, args={label = {nic},type = {scholarly_paper.author}, scholarly_paper.author.email = {nicolas.meric@universite-paris-saclay.fr}, scholarly_paper.author.orcid = {0000-0002-0756-7072}, scholarly_paper.author.affiliation = {Université Paris-Saclay, LMF, France}, scholarly_paper.author.http_site = {}}]Nicolas Méric
\end{isamarkuptext*}\isamarkuptrue%

\begin{isamarkuptext*}
[label = {bu},type = {scholarly_paper.author}, args={label = {bu},type = {scholarly_paper.author}, scholarly_paper.author.email = {burkhart.wolff@universite-paris-saclay.fr}, scholarly_paper.author.affiliation = {Université Paris-Saclay, LMF, France}, scholarly_paper.author.http_site = {}, scholarly_paper.author.orcid = {}}]Burkhart Wolff
\end{isamarkuptext*}\isamarkuptrue%

\begin{isamarkuptext*}
[label = {abs},type = {scholarly_paper.abstract}, args={label = {abs},type = {scholarly_paper.abstract}, scholarly_paper.abstract.keywordlist = {{Ontologies, Formal Documents, Formal Development, Isabelle/HOL, Ontology Mapping, Certification}}}]Isabelle/DOF is an ontology framework on top of Isabelle 
   \cite{brucker.ea:isabelledof:2019,brucker.ea:isabelle-ontologies:2018}. 
   Isabelle/DOF  allows for the formal development of ontologies as well as continuous checking that
   a formal document under development conforms to an underlying ontology. Such a document may 
   contain text and code elements as well as formal Isabelle definitions and proofs.
   Thus, Isabelle/DOF is designed to annotate and trace typed meta-data 
   within formal developments in Isabelle.  

   In this paper we extend  Isabelle/DOF with \emph{invariants} (or: ontological \emph{rules}). Via a reflection 
   mechanism, this allows for efficient run-time checking of abstract properties of formal
   content under evolution. Additionally, invariants have a formal represention in HOL amenable to
   formal proofs over mappings between different ontologies.  
   With this feature widely called \emph{ontology mapping} in the literature, our framework paves the 
   way for advanced uses such as ``semantic'' search and translation. We demonstrate the use of 
   these new features in an extended  ontology used for formal developments targeting CENELEC 
   certifications.
    \vspace{-0.3cm}
\end{isamarkuptext*}\isamarkuptrue%

\begin{isamarkupsection*}
[label = {introheader},type = {scholarly_paper.introduction}, args={label = {introheader},type = {scholarly_paper.introduction}, Isa_COL.text_element.level = {}, Isa_COL.text_element.referentiable = {False}, Isa_COL.text_element.variants = {{STR ''outline'', STR ''document''}}, scholarly_paper.text_section.main_author = {}, scholarly_paper.text_section.fixme_list = {}, Isa_COL.text_element.level = {}}]Introduction
\end{isamarkupsection*}\isamarkuptrue%

\begin{isamarkuptext*}
[label = {introtext},type = {scholarly_paper.introduction}, args={label = {introtext},type = {scholarly_paper.introduction}, Isa_COL.text_element.level = {}, Isa_COL.text_element.referentiable = {False}, Isa_COL.text_element.variants = {{STR ''outline'', STR ''document''}}, scholarly_paper.text_section.main_author = {}, scholarly_paper.text_section.fixme_list = {}, Isa_COL.text_element.level = {}}]\vspace{-0.2cm}
The linking of \emph{formal} and \emph{informal} information is perhaps the most pervasive challenge 
in the digitization of knowledge and its propagation. Unsurprisingly, this problem reappears 
in the libraries with formalized mathematics and engineering such as the Isabelle Archive of 
Formal Proofs \cite{AFP-ref22} (AFP), which passed the impressive numbers of 650 articles, 
written by 420 authors at the beginning of 2022. Together with the AFP, there is also a growing 
body on articles concerned with formal software engineering such as standardized language 
definitions (e.g., \cite{CakeML-AFP,brucker.ea:featherweight:2014}), 
data-structures 
(e.g., \cite{brucker.ea:afp-core-dom:2018,Splay_Tree-AFP}), hardware- models 
(e.g., \cite{SPARCv8-AFP}), 
security-related specifications 
(e.g., \cite{brucker.ea:upf-firewall:2017,Security_Protocol_Refinement-AFP}), 
or operating systems (e. g., \cite{verbeek.ea:formal:2014,klein.ea:comprehensive:2014}).


Still, while the problem of logical consistency
even under system-changes and pervasive theory evolution is technically solved via continuous 
proof-checking, the problem of knowledge retrieval and of linking semi-formal explanations to 
definitions and proofs remains largely open.
% The \emph{knowledge} problem of the increasingly massive \emph{digital information} available
% incites numerous research efforts summarized under the labels ``semantic web'', 
% ``integrated document management'', or any form of advanced ``semantic'' text processing. 
% These technologies are increasingly important  in jurisprudence, medical research and 
% life-sciences in order to tame their respective publication tsunamies. 
The central role in technologies adressing the \emph{knowledge} problem
is played by \emph{document ontologies}, \ie, a machine-readable form
of meta-data attached to document-elements as well as their document discourse. In order 
to make these techniques applicable to \emph{formal theory development},
the following is needed: \vspace{0.2cm}

%
\begin{itemize}%
\item a general mechanism to define and develop \emph{domain-specific} ontologies,

\item ... that should be adapted to entities occurring in formal theories,
\ie, provide built-in support for types, terms, theorems, proofs, etc.,

\item ways to annotate meta-data generated by ontologies to the document elements,
as ``deep'' as possible, together with strong validation checks,

\item a smooth integration into the theory document development process, and

\item ways to relate ontologies and ontology-conform documents along different
ontologies by \emph{ontological mappings} and \emph{data translations}
\footnote{We follow throughout this text the terminology established in 
              \cite{books/daglib/0032976}, pp. 39 ff.}.%
\end{itemize}
\end{isamarkuptext*}\isamarkuptrue%

\begin{isamarkuptext}
\vspace{-0.2cm}
Recently, Isabelle/DOF \cite{brucker.ea:isabelledof:2019,brucker.ea:isabelle-ontologies:2018}
\footnote{The official releases are available at \url{https://zenodo.org/record/6385695}, the
  developer version at \url{https://github.com/logicalhacking/Isabelle_DOF}.} 
has been designed as an Isabelle component that attempts to answer these needs. 
 Isabelle/DOF generates from  ontology definitions directly integrated into Isabelle theories
typed meta-data, that may be annotated to a number of document elements and that were 
validated ``on-the-fly'' during the general continuous type and proof-checking process 
in an IDE (Isabelle/PIDE). Thus, we extend the document-centric view on code, definitions, 
proofs, text-elements, etc., prevailing in the Isabelle system framework.

In more detail, Isabelle/DOF introduces a number of ``ontology aware'' text-elements with analogous 
syntax to standard ones. The difference is a bracket with meta-data of the form:

\begin{isabelle}
\ \ \ \ \ \ \ \ \ \ \isacommand{text{\isacharasterisk}{\kern0pt}}{\isacharbrackleft}{\kern0pt}label{\isacharcolon}{\kern0pt}{\isacharcolon}{\kern0pt}classid{\isacharcomma}{\kern0pt}\ attr\isactrlsub {\isadigit{1}}{\isacharequal}{\kern0pt}E\isactrlsub {\isadigit{1}}{\isacharcomma}{\kern0pt}\ \isacommand{{\isachardot}{\kern0pt}{\isachardot}{\kern0pt}}\isacommand{{\isachardot}{\kern0pt}}\ attr\isactrlsub n{\isacharequal}{\kern0pt}E\isactrlsub n{\isacharbrackright}{\kern0pt}{\isacartoucheopen}\ some\ semi{\isacharminus}{\kern0pt}formal\ text\ {\isacartoucheclose}\isanewline
\ \ \ \ \ \ \ \ \ \ \isacommand{ML{\isacharasterisk}{\kern0pt}}{\isacharbrackleft}{\kern0pt}label{\isacharcolon}{\kern0pt}{\isacharcolon}{\kern0pt}classid{\isacharcomma}{\kern0pt}\ attr\isactrlsub {\isadigit{1}}{\isacharequal}{\kern0pt}E\isactrlsub {\isadigit{1}}{\isacharcomma}{\kern0pt}\ \isacommand{{\isachardot}{\kern0pt}{\isachardot}{\kern0pt}}\isacommand{{\isachardot}{\kern0pt}}\ attr\isactrlsub n{\isacharequal}{\kern0pt}E\isactrlsub n{\isacharbrackright}{\kern0pt}{\isacartoucheopen}\ some\ SML\ code\ {\isacartoucheclose}\isanewline
\ \ \ \ \ \ \ \ \ \ \isacommand{{\isachardot}{\kern0pt}{\isachardot}{\kern0pt}}\isacommand{{\isachardot}{\kern0pt}}
\end{isabelle}
In these Isabelle/DOF elements, a meta-data object is created and associated to it. This 
meta-data can be referenced via its label and used in further computations in text or code.
%; the details will be explained in the subsequent section. 

Admittedly, Isabelle is not the first system that comes into one's mind when writing a scientific 
paper, a book, or a larger technical documentation. However, it has a typesetting system inside 
which is in the tradition of document generation systems such as mkd, Document! X, Doxygen, 
Javadoc, etc., and which embed formal content such as formula pretty-prints into semi-formal text 
or code. The analogous mechanism the Isabelle system provides is a machine-checked macro 
called \emph{antiquotation} that depends on the logical context of the document element.

With standard Isabelle antiquotations, for example, the following text element
of the integrated source will appear  in Isabelle/PIDE as follows:

\begin{isabelle}
\ \ \ \ \ \ \ \ \ \ \isacommand{text}{\isacartoucheopen}\ According\ to\ the\ reflexivity\ axiom\ {\isacharat}{\kern0pt}{\isacharbraceleft}{\kern0pt}thm\ refl{\isacharbraceright}{\kern0pt}{\isacharcomma}{\kern0pt}\ we\ obtain\ in\ {\isasymGamma}\isanewline
\ \ \ \ \ \ \ \ \ \ \ \ \ \ \ \ \ for\ {\isacharat}{\kern0pt}{\isacharbraceleft}{\kern0pt}term\ {\isachardoublequote}{\kern0pt}fac\ {\isadigit{5}}{\isachardoublequote}{\kern0pt}{\isacharbraceright}{\kern0pt}\ the\ result\ {\isacharat}{\kern0pt}{\isacharbraceleft}{\kern0pt}value\ {\isachardoublequote}{\kern0pt}fac\ {\isadigit{5}}{\isachardoublequote}{\kern0pt}{\isacharbraceright}{\kern0pt}{\isachardot}{\kern0pt}{\isacartoucheclose}
\end{isabelle}
In the corresponding generated LaTeX or HTML output, this looks like this:

\begin{isabelle}
\ \ \ \ \ \ \ \ \ \ \ \ \ \ \ \ \ According\ to\ the\ reflexivity\ axiom\ {\isasymopen}x\ {\isacharequal}{\kern0pt}\ x{\isasymclose}{\isacharcomma}{\kern0pt}\ we\ obtain\ in\ {\isasymGamma}\ \isanewline
\ \ \ \ \ \ \ \ \ \ \ \ \ \ \ \ \ \ for\ {\isasymopen}fac\ {\isadigit{5}}{\isasymclose}\ the\ result\ {\isasymopen}{\isadigit{1}}{\isadigit{2}}{\isadigit{0}}{\isasymclose}{\isachardot}{\kern0pt}
\end{isabelle}
where the meta-texts \isa{{\isacharat}{\kern0pt}{\isacharbraceleft}{\kern0pt}thm\ refl{\isacharbraceright}{\kern0pt}} (``give the presentation of theorem `refl'\,\!''), 
\isa{{\isacharat}{\kern0pt}{\isacharbraceleft}{\kern0pt}term\ {\isachardoublequote}{\kern0pt}fac\ {\isadigit{5}}{\isachardoublequote}{\kern0pt}{\isacharbraceright}{\kern0pt}} (``parse and type-check `fac 5' in the previous logical context'')
and \isa{{\isacharat}{\kern0pt}{\isacharbraceleft}{\kern0pt}value\ {\isachardoublequote}{\kern0pt}fac\ {\isadigit{5}}{\isachardoublequote}{\kern0pt}{\isacharbraceright}{\kern0pt}} (``compile and execute `fac 5' according to its
definitions'') are built-in antiquotations in HOL. 

One distinguishing feature of Isabelle/DOF is that specific antiquotations \emph{were generated from
an ontology} rather than being hard-coded into the Isabelle system infrastructure.
\end{isamarkuptext}\isamarkuptrue%

\begin{isamarkuptext}
As novel contribution, this work extends prior versions of Isabelle/DOF by 

%
\begin{enumerate}%
\item support of antiquotations in a new class of contexts, namely \emph{term contexts} 
(rather than SML code or semi-formal text). Thus, annotations generated
from  Isabelle/DOF may also occur in \isa{{\isasymlambda}}-terms used to denote meta-data. 

\item formal, machine-checked invariants on meta-data, which correspond to the concept of 
``rules'' in OWL~ \cite{OWL2014} or ``constraints'' in UML, and which can be specified in 
common HOL \isa{{\isasymlambda}}-term syntax.%
\end{enumerate}
\end{isamarkuptext}\isamarkuptrue%

\begin{isamarkuptext}
\noindent For example, the Isabelle/DOF command evaluating the HOL-expression:

\begin{isabelle}
\ \ \ \ \ \ \ \ \ \ \isacommand{value{\isacharasterisk}{\kern0pt}}{\isacharbrackleft}{\kern0pt}ass{\isacharcolon}{\kern0pt}{\isacharcolon}{\kern0pt}Assertion{\isacharcomma}{\kern0pt}\ relvce{\isacharequal}{\kern0pt}{\isadigit{4}}{\isacharcolon}{\kern0pt}{\isacharcolon}{\kern0pt}int{\isacharbrackright}{\kern0pt}\ \isanewline
\ \ \ \ \ \ \ \ \ \ \ \ \ \ \ \ \ \ {\isacartoucheopen}filter\ {\isacharparenleft}{\kern0pt}{\isasymlambda}\ {\isasymsigma}{\isachardot}{\kern0pt}\ relvce\ {\isasymsigma}\ {\isachargreater}{\kern0pt}\ {\isadigit{2}}{\isacharparenright}{\kern0pt}\ {\isacharat}{\kern0pt}{\isacharbraceleft}{\kern0pt}Assertion{\isacharminus}{\kern0pt}instances{\isacharbraceright}{\kern0pt}{\isacartoucheclose}
\end{isabelle}
where Isabelle/DOF command \isa{value{\isacharasterisk}{\kern0pt}} type-checks, expands in an own validation phase
the \isa{Assertion{\isacharminus}{\kern0pt}instances}-term antiquotation, and evaluates the resulting HOL expression
above. Assuming an ontology providing the class \isa{Assertion} having at least the
integer attribute \isa{relvce}, the command finally creates an instance of \isa{Assertion} and 
binds this to label \isa{ass}, while setting its \isa{relvce} to 4.

Beyond the gain of expressivity in Isabelle/DOF ontologies, term-anti\-quotations pave the way 
for advanced queries of elements inside an integrated source, and invariants 
allow for formal proofs over the relations/translations of ontologies and ontology-instances.
The latter question raised scientific interest under the label ``ontology mapping'' for 
which we therefore present a formal solution. To sum up, we completed Isabelle/DOF to
a fairly rich ontology language oriented to interactive theorem proving (ITP) systems, 
which is a concrete proposal for formal development projects targeting a certification, 
for technical documentation, for books with a high amount of machine-checked formal content 
or for mathematical libraries such as the AFP.
\end{isamarkuptext}\isamarkuptrue%

\begin{isamarkupsection*}
[label = {bgrnd},type = {scholarly_paper.background}, args={label = {bgrnd},type = {scholarly_paper.background}, scholarly_paper.text_section.main_author = {@{docitem ''bu''}}, Isa_COL.text_element.level = {}, Isa_COL.text_element.referentiable = {False}, Isa_COL.text_element.variants = {{STR ''outline'', STR ''document''}}, scholarly_paper.text_section.fixme_list = {}, Isa_COL.text_element.level = {}}]Background
\end{isamarkupsection*}\isamarkuptrue%

\begin{isamarkupsubsection*}
[label = {bgrnd_isadof},type = {scholarly_paper.background}, args={label = {bgrnd_isadof},type = {scholarly_paper.background}, Isa_COL.text_element.level = {}, Isa_COL.text_element.referentiable = {False}, Isa_COL.text_element.variants = {{STR ''outline'', STR ''document''}}, scholarly_paper.text_section.main_author = {}, scholarly_paper.text_section.fixme_list = {}, Isa_COL.text_element.level = {}}]The Isabelle/DOF Framework
\end{isamarkupsubsection*}\isamarkuptrue%

\begin{isamarkuptext}
Isabelle/DOF~\cite{brucker.ea:isabelle-ontologies:2018,brucker.ea:isabelledof:2019} 
  is a document ontology framework that extends Isabelle/HOL.
  Isabelle/DOF offers basically two things: a language called Ontology Definition Language (ODL)
  to \emph{specify} a formal ontology,
  and ways to \emph{annotate} an integrated document written in Isabelle/HOL with the specified
  meta-data. Additionally, Isabelle/DOF generates from an ontology a family of 
  \emph{anti\-quotations} allowing to specify  machine-checked links between ODL entities. 
%  Unlike UML, however, Isabelle/DOF allows for integrated documents with informal and formal elements
%  including the necessary management of logical contexts. 

  The perhaps most attractive aspect of Isabelle/DOF is its deep integration into the IDE of Isabelle 
  (Isabelle/PIDE), which allows a hypertext-like navigation as well as fast user-feedback
  during development and evolution of the integrated source. This includes rich editing support, 
  including on-the-fly semantics checks, hinting, or auto-completion. 
  Isabelle/DOF supports LaTeX-based document generation as well as ontology-aware ``views'' on 
  the integrated document, \ie, specific versions of generated PDF addressing, \eg,
  different stake-holders.
\end{isamarkuptext}\isamarkuptrue%

\begin{isamarkupsubsection*}
[label = {bgrnd_ODL},type = {scholarly_paper.background}, args={label = {bgrnd_ODL},type = {scholarly_paper.background}, Isa_COL.text_element.level = {}, Isa_COL.text_element.referentiable = {False}, Isa_COL.text_element.variants = {{STR ''outline'', STR ''document''}}, scholarly_paper.text_section.main_author = {}, scholarly_paper.text_section.fixme_list = {}, Isa_COL.text_element.level = {}}]A Guided Tour through ODL
\end{isamarkupsubsection*}\isamarkuptrue%

\begin{isamarkuptext}
Isabelle/DOF provides a strongly typed ODL that provides the usual 
concepts of ontologies such as

%
\begin{itemize}%
\item \emph{document class} (using the \isa{\isacommand{doc{\isacharunderscore}{\kern0pt}class}} keyword) that describes a concept,

\item \emph{attributes} specific to document classes (attributes might be initialized with default 
values), and

\item a special link, the reference to a super-class,
establishes an \emph{is-a} relation between classes.
%  classes may refer to other classes via a regular expression in an optional \emph{where} clause 
%    (a class with a where clause is called \emph{monitor}).%
\end{itemize}
\end{isamarkuptext}\isamarkuptrue%

\begin{isamarkuptext}
\vspace{-0.2cm} The types of attributes are HOL-types. Thus, ODL can refer to any predefined type 
  from the HOL library, \eg, \isa{string}, \isa{int} as well as parameterized types, \eg, 
  \isa{option},  \isa{list}. As a consequence of the Isabelle document model, ODL definitions 
  may be arbitrarily mixed with standard HOL type definitions. Document class definitions are 
  HOL-types, allowing for formal \emph{links} to and between ontological concepts. For example, the 
  basic concept of requirements from CENELEC 50128~\cite{bsi:50128:2014} is captured in ODL as 
  follows:
  
\begin{isabelle}
\ \ \ \ \ \ \ \ \ \ \isacommand{doc{\isacharunderscore}{\kern0pt}class}\ requirement\ {\isacharequal}{\kern0pt}\ text{\isacharunderscore}{\kern0pt}element\ {\isacharplus}{\kern0pt}\ \ \ \isanewline
\ \ \ \ \ \ \ \ \ \ \ \ \ \ \ \ \ \ \ \ long{\isacharunderscore}{\kern0pt}name\ \ \ {\isacharcolon}{\kern0pt}{\isacharcolon}{\kern0pt}{\isachardoublequoteopen}string\ option{\isachardoublequoteclose}\ \ \ \isanewline
\ \ \ \ \ \ \ \ \ \ \ \ \ \ \ \ \ \ \ \ is{\isacharunderscore}{\kern0pt}concerned{\isacharcolon}{\kern0pt}{\isacharcolon}{\kern0pt}{\isachardoublequoteopen}role\ set{\isachardoublequoteclose}\ \ \ \ \ \ \ \ \ 
\end{isabelle}
  This ODL class definition maybe part of one or more Isabelle theory-files capturing the entire
  ontology definition. Isabelle's session management allows for pre-compiling them before being 
  imported in the actual target document.
\end{isamarkuptext}\isamarkuptrue%

\begin{isamarkupsideUNDERSCOREbyUNDERSCOREsideUNDERSCOREfigure*}
[label = {text-elements},type = {Isa_COL.side_by_side_figure}, args={label = {text-elements},type = {Isa_COL.side_by_side_figure}, Isa_COL.side_by_side_figure.anchor = {fig-Req-Def-ex}, Isa_COL.side_by_side_figure.caption = {A Text-Element as Requirement.}, Isa_COL.figure.relative_width = {48}, Isa_COL.figure.src = {figures/Req-Def-ex}, Isa_COL.side_by_side_figure.anchor2 = {fig-Req-Appl-ex}, Isa_COL.side_by_side_figure.caption2 = {Referencing a Requirement.}, Isa_COL.side_by_side_figure.relative_width2 = {48}, Isa_COL.side_by_side_figure.src2 = {figures/Req-Appl-ex}, Isa_COL.figure.spawn_columns = {True}}]Referencing a Requirement.
\end{isamarkupsideUNDERSCOREbyUNDERSCOREsideUNDERSCOREfigure*}\isamarkuptrue%

\begin{isamarkuptext}
\csname isaDof.ref\endcsname[type={Isa_COL.figure}]     {text-elements} shows an ontological annotation of a requirement and its referencing
  via an antiquotation \isa{{\isacharat}{\kern0pt}\isacommand{{\isacharbraceleft}{\kern0pt}}requirement\ {\isacartoucheopen}req{\isadigit{1}}{\isacartoucheclose}\isacommand{{\isacharbraceright}{\kern0pt}}}; the latter is generated from the above 
  class definition. Undefined or ill-typed references were rejected, the high-lighting displays 
  the hyperlinking which is activated on a click. The class-definition of \isa{requirement} and its 
  documentation is also revisited via one activation click.
\end{isamarkuptext}\isamarkuptrue%

\begin{isamarkuptext}
\vspace{-0.1cm} \noindent Isabelle/HOL supports records at the level of terms and 
types. The notation for terms and types is as follows: \vspace{-0.2cm}

%
\begin{itemize}%
\item record terms \isa{{\isasymlparr}x\ {\isacharequal}{\kern0pt}\ a{\isacharcomma}{\kern0pt}\ y\ {\isacharequal}{\kern0pt}\ b{\isasymrparr}} and corresponding record types \isa{{\isasymlparr}x{\isacharcolon}{\kern0pt}{\isacharcolon}{\kern0pt}A{\isacharcomma}{\kern0pt}\ y{\isacharcolon}{\kern0pt}{\isacharcolon}{\kern0pt}B{\isasymrparr}},

\item the resulting selectors are written \isa{x\ r}, \isa{y\ r}.%
\end{itemize}
\end{isamarkuptext}\isamarkuptrue%

\begin{isamarkuptext}
\noindent In fact, \isa{\isacommand{onto{\isacharunderscore}{\kern0pt}class}}es and the logically equivalent  \isa{\isacommand{doc{\isacharunderscore}{\kern0pt}class}}es were
represented by \emph{extensible} record types and instances thereof by HOL terms 
(see \cite{brucker.ea:isabelledof:2019} for details.).  
Invariants of an \isa{\isacommand{onto{\isacharunderscore}{\kern0pt}class}} are just predicates over extensible record
types and were applied to subclasses.
\end{isamarkuptext}\isamarkuptrue%
%
\isadelimdocument
%
\endisadelimdocument
%
\isatagdocument
%
\isamarkupsubsection{Term-Evaluations in Isabelle%
}
\isamarkuptrue%
%
\endisatagdocument
{\isafolddocument}%
%
\isadelimdocument
%
\endisadelimdocument

\begin{isamarkuptext}
Besides the powerful, but relatively slow Isabelle rewriting-based proof method,
there are  two other techniques for term evaluation: \vspace{-0.2cm}

%
\begin{itemize}%
\item evaluation via reflection \cite{HaftmannN10} (\isa{eval}), and

\item normalization by evaluation \cite{AehligHN12} (\isa{nbe}).%
\end{itemize}
\end{isamarkuptext}\isamarkuptrue%

\begin{isamarkuptext}
\noindent  The former is based on a nearly one-to-one compilation of HOL-level datatype specifications 
and function definitions into SML datatypes and functions.
The latter technique --- allowing for free variables in terms --- uses a generic data-universe
enriched by explicit variables. Both techniques are several orders of magnitude faster 
than standard rewriting. Isabelle/DOF uses both to generate code that evaluates invariant and data-integrity 
checks on-the-fly during editing. For all examples in our library, this form of runtime-testing
 is sufficiently fast to remain unnoticed by the user.
\end{isamarkuptext}\isamarkuptrue%

\begin{isamarkupsection*}
[label = {invariants},type = {scholarly_paper.technical}, args={label = {invariants},type = {scholarly_paper.technical}, scholarly_paper.text_section.main_author = {@{docitem ''nic''}}, Isa_COL.text_element.level = {}, Isa_COL.text_element.referentiable = {False}, Isa_COL.text_element.variants = {{STR ''outline'', STR ''document''}}, scholarly_paper.text_section.fixme_list = {}, Isa_COL.text_element.level = {}, scholarly_paper.technical.definition_list = {}, scholarly_paper.technical.status = {description}}]Term-Context Support, Invariants and Queries in DOF
\end{isamarkupsection*}\isamarkuptrue%

\begin{isamarkuptext}
To offer a smooth integration into the \emph{formal} theory development process,
Isabelle/HOL should be able to dynamically interpret the source document.
But the specific antiquotations introduced by Isabelle/DOF are not directly recognized
by Isabelle/HOL, and the process of term checking and evaluation must be enriched.
Previous works~\cite{brucker.ea:isabelle-ontologies:2018,brucker.ea:isabelledof:2019}
added a validation step for the SML and semi-formal text contexts.
To support Isabelle/DOF antiquotations in the term contexts, the validation step should
be improved and a new step, which we call \emph{elaboration} must be added to allow
these antiquotations in \isa{{\isasymlambda}}-terms.
The resulting process encompasses the following steps:

%
\begin{itemize}%
\item Parsing of the term which represents the object in HOL,

\item Typeinference/Typechecking of the term,

\item Ontological validation of the term: the meta-data of term antiquotations is 
parsed and checked in the logical context,

\item Elaboration of term antiquotations: depending of the antiquotation specific
elaboration function, the antiquotations containing references were replaced,
for example, by the object they refer to in the logical context,

\item Generation of markup information for the Isabelle/PIDE, and

\item Code generation:

%
\begin{itemize}%
\item Evaluation of HOL expressions with ontological annotations,

\item Generation of ontological invariants processed simultaneously
with the generation of the document (a theory in HOL).%
\end{itemize}%
\end{itemize}

Isabelle/HOL provides inspection commands to type-check (the command \isa{\isacommand{term}})
and to evaluate a term (the command \isa{\isacommand{value}}).
We provide the equivalent commands, respectively \isa{\isacommand{term{\isacharasterisk}{\kern0pt}}} and \isa{\isacommand{value{\isacharasterisk}{\kern0pt}}}, which 
additionally support a validation and elaboration phase.
A variant of  \isa{\isacommand{value{\isacharasterisk}{\kern0pt}}} is  \isa{\isacommand{assert{\isacharasterisk}{\kern0pt}}}, which additionally checks
that the term-evaluation results in \isa{True}.
Note that term antiquotations are admitted in all Isabelle/DOF commands, not just
\isa{\isacommand{term{\isacharasterisk}{\kern0pt}}}, \isa{\isacommand{value{\isacharasterisk}{\kern0pt}}} etc. 
\end{isamarkuptext}\isamarkuptrue%

\begin{isamarkuptext}
If we take back the example ontology for mathematical papers
  of~\cite{brucker.ea:isabelledof:2019} shown in \autoref{fig-ontology-example}
\begin{figure}

\begin{isarbox}

\begin{isabelle}
\isacommand{datatype}\ kind\ {\isacharequal}{\kern0pt}\ expert{\isacharunderscore}{\kern0pt}opinion\ {\isacharbar}{\kern0pt}\ argument\ {\isacharbar}{\kern0pt}\ {\isachardoublequoteopen}proof{\isachardoublequoteclose}\isanewline
\isanewline
\isacommand{doc{\isacharunderscore}{\kern0pt}class}\ myauthor\ {\isacharequal}{\kern0pt}\isanewline
\ \ email\ {\isacharcolon}{\kern0pt}{\isacharcolon}{\kern0pt}\ {\isachardoublequoteopen}string{\isachardoublequoteclose}\ {\isacharless}{\kern0pt}{\isacharequal}{\kern0pt}\ {\isachardoublequoteopen}{\isacharprime}{\kern0pt}{\isacharprime}{\kern0pt}{\isacharprime}{\kern0pt}{\isacharprime}{\kern0pt}{\isachardoublequoteclose}\isanewline
\isacommand{doc{\isacharunderscore}{\kern0pt}class}\ mytext{\isacharunderscore}{\kern0pt}section\ {\isacharequal}{\kern0pt}\isanewline
\ \ authored{\isacharunderscore}{\kern0pt}by\ {\isacharcolon}{\kern0pt}{\isacharcolon}{\kern0pt}\ {\isachardoublequoteopen}myauthor\ set{\isachardoublequoteclose}\ {\isacharless}{\kern0pt}{\isacharequal}{\kern0pt}\ {\isachardoublequoteopen}{\isacharbraceleft}{\kern0pt}{\isacharbraceright}{\kern0pt}{\isachardoublequoteclose}\isanewline
\ \ level\ {\isacharcolon}{\kern0pt}{\isacharcolon}{\kern0pt}\ {\isachardoublequoteopen}int\ option{\isachardoublequoteclose}\ {\isacharless}{\kern0pt}{\isacharequal}{\kern0pt}\ {\isachardoublequoteopen}None{\isachardoublequoteclose}\isanewline
\isacommand{doc{\isacharunderscore}{\kern0pt}class}\ myintro\ {\isacharequal}{\kern0pt}\ mytext{\isacharunderscore}{\kern0pt}section\ {\isacharplus}{\kern0pt}\isanewline
\ \ authored{\isacharunderscore}{\kern0pt}by\ {\isacharcolon}{\kern0pt}{\isacharcolon}{\kern0pt}\ {\isachardoublequoteopen}myauthor\ set{\isachardoublequoteclose}\ \ {\isacharless}{\kern0pt}{\isacharequal}{\kern0pt}\ {\isachardoublequoteopen}UNIV{\isachardoublequoteclose}\ \isanewline
\ \ uses\ {\isacharcolon}{\kern0pt}{\isacharcolon}{\kern0pt}\ {\isachardoublequoteopen}string\ set{\isachardoublequoteclose}\isanewline
\ \ \isakeyword{invariant}\ author{\isacharunderscore}{\kern0pt}set\ {\isacharcolon}{\kern0pt}{\isacharcolon}{\kern0pt}\ {\isachardoublequoteopen}authored{\isacharunderscore}{\kern0pt}by\ {\isasymsigma}\ {\isasymnoteq}\ {\isacharbraceleft}{\kern0pt}{\isacharbraceright}{\kern0pt}{\isachardoublequoteclose}\isanewline
\ \ \isakeyword{and}\ force{\isacharunderscore}{\kern0pt}level\ {\isacharcolon}{\kern0pt}{\isacharcolon}{\kern0pt}\ {\isachardoublequoteopen}the\ {\isacharparenleft}{\kern0pt}level\ {\isasymsigma}{\isacharparenright}{\kern0pt}\ {\isachargreater}{\kern0pt}\ {\isadigit{1}}{\isachardoublequoteclose}\isanewline
\isacommand{doc{\isacharunderscore}{\kern0pt}class}\ myclaim\ {\isacharequal}{\kern0pt}\ myintro\ {\isacharplus}{\kern0pt}\isanewline
\ \ based{\isacharunderscore}{\kern0pt}on\ {\isacharcolon}{\kern0pt}{\isacharcolon}{\kern0pt}\ {\isachardoublequoteopen}string\ list{\isachardoublequoteclose}\isanewline
\isacommand{doc{\isacharunderscore}{\kern0pt}class}\ mytechnical\ {\isacharequal}{\kern0pt}\ mytext{\isacharunderscore}{\kern0pt}section\ {\isacharplus}{\kern0pt}\isanewline
\ \ formal{\isacharunderscore}{\kern0pt}results\ {\isacharcolon}{\kern0pt}{\isacharcolon}{\kern0pt}\ {\isachardoublequoteopen}thm\ list{\isachardoublequoteclose}\ \isanewline
\isacommand{doc{\isacharunderscore}{\kern0pt}class}\ myresult\ {\isacharequal}{\kern0pt}\ mytechnical\ {\isacharplus}{\kern0pt}\isanewline
\ \ evidence\ {\isacharcolon}{\kern0pt}{\isacharcolon}{\kern0pt}\ kind\isanewline
\ \ property\ {\isacharcolon}{\kern0pt}{\isacharcolon}{\kern0pt}\ {\isachardoublequoteopen}thm\ list{\isachardoublequoteclose}\ {\isacharless}{\kern0pt}{\isacharequal}{\kern0pt}\ {\isachardoublequoteopen}{\isacharbrackleft}{\kern0pt}{\isacharbrackright}{\kern0pt}{\isachardoublequoteclose}\isanewline
\ \ \isakeyword{invariant}\ has{\isacharunderscore}{\kern0pt}property\ {\isacharcolon}{\kern0pt}{\isacharcolon}{\kern0pt}\ {\isachardoublequoteopen}evidence\ {\isasymsigma}\ {\isacharequal}{\kern0pt}\ proof\ {\isasymlongleftrightarrow}\ property\ {\isasymsigma}\ {\isasymnoteq}\ {\isacharbrackleft}{\kern0pt}{\isacharbrackright}{\kern0pt}{\isachardoublequoteclose}\isanewline
\isacommand{doc{\isacharunderscore}{\kern0pt}class}\ myconclusion\ {\isacharequal}{\kern0pt}\ mytext{\isacharunderscore}{\kern0pt}section\ {\isacharplus}{\kern0pt}\isanewline
\ \ establish\ {\isacharcolon}{\kern0pt}{\isacharcolon}{\kern0pt}\ {\isachardoublequoteopen}{\isacharparenleft}{\kern0pt}myclaim\ {\isasymtimes}\ myresult{\isacharparenright}{\kern0pt}\ set{\isachardoublequoteclose}\isanewline
\ \ \isakeyword{invariant}\ establish{\isacharunderscore}{\kern0pt}defined\ {\isacharcolon}{\kern0pt}{\isacharcolon}{\kern0pt}\ {\isachardoublequoteopen}{\isasymforall}\ x{\isachardot}{\kern0pt}\ x\ {\isasymin}\ Domain\ {\isacharparenleft}{\kern0pt}establish\ {\isasymsigma}{\isacharparenright}{\kern0pt}\isanewline
\ \ \ \ \ \ \ \ \ \ \ \ \ \ \ \ \ \ \ \ \ \ {\isasymlongrightarrow}\ {\isacharparenleft}{\kern0pt}{\isasymexists}\ y\ {\isasymin}\ Range\ {\isacharparenleft}{\kern0pt}establish\ {\isasymsigma}{\isacharparenright}{\kern0pt}{\isachardot}{\kern0pt}\ {\isacharparenleft}{\kern0pt}x{\isacharcomma}{\kern0pt}\ y{\isacharparenright}{\kern0pt}\ {\isasymin}\ establish\ {\isasymsigma}{\isacharparenright}{\kern0pt}{\isachardoublequoteclose}
\end{isabelle}
\end{isarbox}
\caption{Excerpt of an Example Ontology for mathematical Papers.}
\label{fig-ontology-example}
\end{figure}
  we can define some class instances for this ontology with the \isa{\isacommand{text{\isacharasterisk}{\kern0pt}}} command,
  as in \autoref{fig-instances-example}.
\begin{figure}

\begin{isarbox}

\begin{isabelle}
\isacommand{text{\isacharasterisk}{\kern0pt}}{\isacharbrackleft}{\kern0pt}church{\isacharcolon}{\kern0pt}{\isacharcolon}{\kern0pt}myauthor{\isacharcomma}{\kern0pt}\ email{\isacharequal}{\kern0pt}{\isachardoublequoteopen}{\isasymopen}church{\isacharat}{\kern0pt}lambda{\isachardot}{\kern0pt}org{\isasymclose}{\isachardoublequoteclose}{\isacharbrackright}{\kern0pt}{\isacartoucheopen}{\isacartoucheclose}\isanewline
\isacommand{text{\isacharasterisk}{\kern0pt}}{\isacharbrackleft}{\kern0pt}proof{\isadigit{1}}{\isacharcolon}{\kern0pt}{\isacharcolon}{\kern0pt}myresult{\isacharcomma}{\kern0pt}\ evidence{\isacharequal}{\kern0pt}{\isachardoublequoteopen}proof{\isachardoublequoteclose}{\isacharcomma}{\kern0pt}\ property{\isacharequal}{\kern0pt}{\isachardoublequoteopen}{\isacharbrackleft}{\kern0pt}{\isacharat}{\kern0pt}{\isacharbraceleft}{\kern0pt}thm\ {\isasymopen}HOL{\isachardot}{\kern0pt}refl{\isasymclose}{\isacharbraceright}{\kern0pt}{\isacharbrackright}{\kern0pt}{\isachardoublequoteclose}{\isacharbrackright}{\kern0pt}{\isacartoucheopen}{\isacartoucheclose}\isanewline
\isacommand{text{\isacharasterisk}{\kern0pt}}{\isacharbrackleft}{\kern0pt}proof{\isadigit{2}}{\isacharcolon}{\kern0pt}{\isacharcolon}{\kern0pt}myresult{\isacharcomma}{\kern0pt}\ evidence{\isacharequal}{\kern0pt}{\isachardoublequoteopen}proof{\isachardoublequoteclose}{\isacharcomma}{\kern0pt}\ property{\isacharequal}{\kern0pt}{\isachardoublequoteopen}{\isacharbrackleft}{\kern0pt}{\isacharat}{\kern0pt}{\isacharbraceleft}{\kern0pt}thm\ {\isasymopen}HOL{\isachardot}{\kern0pt}sym{\isasymclose}{\isacharbraceright}{\kern0pt}{\isacharbrackright}{\kern0pt}{\isachardoublequoteclose}{\isacharbrackright}{\kern0pt}{\isacartoucheopen}{\isacartoucheclose}\isanewline
\isacommand{text{\isacharasterisk}{\kern0pt}}{\isacharbrackleft}{\kern0pt}intro{\isadigit{1}}{\isacharcolon}{\kern0pt}{\isacharcolon}{\kern0pt}myintro{\isacharcomma}{\kern0pt}\ authored{\isacharunderscore}{\kern0pt}by{\isacharequal}{\kern0pt}{\isachardoublequoteopen}{\isacharbraceleft}{\kern0pt}{\isacharat}{\kern0pt}{\isacharbraceleft}{\kern0pt}myauthor\ {\isasymopen}church{\isasymclose}{\isacharbraceright}{\kern0pt}{\isacharbraceright}{\kern0pt}{\isachardoublequoteclose}{\isacharcomma}{\kern0pt}\ level{\isacharequal}{\kern0pt}{\isachardoublequoteopen}Some\ {\isadigit{0}}{\isachardoublequoteclose}{\isacharbrackright}{\kern0pt}{\isacartoucheopen}{\isacartoucheclose}\isanewline
\isacommand{text{\isacharasterisk}{\kern0pt}}{\isacharbrackleft}{\kern0pt}intro{\isadigit{2}}{\isacharcolon}{\kern0pt}{\isacharcolon}{\kern0pt}myintro{\isacharcomma}{\kern0pt}\ authored{\isacharunderscore}{\kern0pt}by{\isacharequal}{\kern0pt}{\isachardoublequoteopen}{\isacharbraceleft}{\kern0pt}{\isacharat}{\kern0pt}{\isacharbraceleft}{\kern0pt}myauthor\ {\isasymopen}church{\isasymclose}{\isacharbraceright}{\kern0pt}{\isacharbraceright}{\kern0pt}{\isachardoublequoteclose}{\isacharcomma}{\kern0pt}\ level{\isacharequal}{\kern0pt}{\isachardoublequoteopen}Some\ {\isadigit{2}}{\isachardoublequoteclose}{\isacharbrackright}{\kern0pt}{\isacartoucheopen}{\isacartoucheclose}\isanewline
\isacommand{text{\isacharasterisk}{\kern0pt}}{\isacharbrackleft}{\kern0pt}claimNotion{\isacharcolon}{\kern0pt}{\isacharcolon}{\kern0pt}myclaim{\isacharcomma}{\kern0pt}\ authored{\isacharunderscore}{\kern0pt}by{\isacharequal}{\kern0pt}{\isachardoublequoteopen}{\isacharbraceleft}{\kern0pt}{\isacharat}{\kern0pt}{\isacharbraceleft}{\kern0pt}myauthor\ {\isasymopen}church{\isasymclose}{\isacharbraceright}{\kern0pt}{\isacharbraceright}{\kern0pt}{\isachardoublequoteclose}\isanewline
\ \ \ \ \ \ {\isacharcomma}{\kern0pt}\ based{\isacharunderscore}{\kern0pt}on{\isacharequal}{\kern0pt}{\isachardoublequoteopen}{\isacharbrackleft}{\kern0pt}{\isasymopen}Notion{\isadigit{1}}{\isasymclose}{\isacharcomma}{\kern0pt}{\isasymopen}Notion{\isadigit{2}}{\isasymclose}{\isacharbrackright}{\kern0pt}{\isachardoublequoteclose}{\isacharcomma}{\kern0pt}\ level{\isacharequal}{\kern0pt}{\isachardoublequoteopen}Some\ {\isadigit{0}}{\isachardoublequoteclose}{\isacharbrackright}{\kern0pt}{\isacartoucheopen}{\isacartoucheclose}
\end{isabelle}
\end{isarbox}
\caption{Some Instances of the Classes of the Ontology of \autoref{fig-ontology-example}.}
\label{fig-instances-example}
\end{figure}
  In the instance \isa{intro{\isadigit{1}}}, the term antiquotation  \isa{{\isacharat}{\kern0pt}\isacommand{{\isacharbraceleft}{\kern0pt}}myauthor\ {\isacartoucheopen}church{\isacartoucheclose}\isacommand{{\isacharbraceright}{\kern0pt}}},
  or its equivalent notation \isa{{\isacharat}{\kern0pt}{\isacharbraceleft}{\kern0pt}myauthor\ {\isacharprime}{\kern0pt}{\isacharprime}{\kern0pt}church{\isacharprime}{\kern0pt}{\isacharprime}{\kern0pt}{\isacharbraceright}{\kern0pt}}, denotes
  the instance \isa{church} of the class \isa{myauthor},
  where \isa{church} is a HOL-string.
  One can now reference a class instance in a \isa{\isacommand{term{\isacharasterisk}{\kern0pt}}} command.
  In the command  \isa{\isacommand{term{\isacharasterisk}{\kern0pt}}{\isacartoucheopen}{\isacharat}{\kern0pt}{\isacharbraceleft}{\kern0pt}myauthor\ {\isasymopen}church{\isasymclose}{\isacharbraceright}{\kern0pt}{\isacartoucheclose}}
  the term \isa{{\isacharat}{\kern0pt}{\isacharbraceleft}{\kern0pt}myauthor\ {\isacharprime}{\kern0pt}{\isacharprime}{\kern0pt}church{\isacharprime}{\kern0pt}{\isacharprime}{\kern0pt}{\isacharbraceright}{\kern0pt}} is type-checked, \ie, the command \isa{\isacommand{term{\isacharasterisk}{\kern0pt}}} checks that
  \isa{church} references a term of type \isa{myauthor} against the global context
  (see \csname isaDof.ref\endcsname[type={Isa_COL.side_by_side_figure}]     {type-checking-example}).
\end{isamarkuptext}\isamarkuptrue%

\begin{isamarkupsideUNDERSCOREbyUNDERSCOREsideUNDERSCOREfigure*}
[label = {type-checking-example},type = {Isa_COL.side_by_side_figure}, args={label = {type-checking-example},type = {Isa_COL.side_by_side_figure}, Isa_COL.side_by_side_figure.anchor = {fig-term-type-checking-ex}, Isa_COL.side_by_side_figure.caption = {Church is an existing Instance.}, Isa_COL.figure.relative_width = {48}, Isa_COL.figure.src = {figures/term-context-checking-example}, Isa_COL.side_by_side_figure.anchor2 = {fig-term-type-checking-failed-ex}, Isa_COL.side_by_side_figure.caption2 = {The Churche Instance is not defined.}, Isa_COL.side_by_side_figure.relative_width2 = {48}, Isa_COL.side_by_side_figure.src2 = {figures/term-context-failed-checking-example}, Isa_COL.figure.spawn_columns = {True}}]Type-Checking of Antiquotations in a Term-Context.
\end{isamarkupsideUNDERSCOREbyUNDERSCOREsideUNDERSCOREfigure*}\isamarkuptrue%

\begin{isamarkuptext}
The command \isa{\isacommand{value{\isacharasterisk}{\kern0pt}}{\isacartoucheopen}email\ {\isacharat}{\kern0pt}{\isacharbraceleft}{\kern0pt}author\ {\isasymopen}church{\isasymclose}{\isacharbraceright}{\kern0pt}{\isacartoucheclose}}
  validates \isa{{\isacharat}{\kern0pt}{\isacharbraceleft}{\kern0pt}myauthor\ {\isacharprime}{\kern0pt}{\isacharprime}{\kern0pt}church{\isacharprime}{\kern0pt}{\isacharprime}{\kern0pt}{\isacharbraceright}{\kern0pt}}
  and returns the attribute-value of \isa{myauthor{\isachardot}{\kern0pt}email} for the \isa{church} instance,
  \ie the HOL string \isa{{\isacharprime}{\kern0pt}{\isacharprime}{\kern0pt}church{\isacharat}{\kern0pt}lambda{\isachardot}{\kern0pt}org{\isacharprime}{\kern0pt}{\isacharprime}{\kern0pt}}
  (see \csname isaDof.ref\endcsname[type={Isa_COL.side_by_side_figure}]     {evaluation-example}).
\end{isamarkuptext}\isamarkuptrue%

\begin{isamarkupsideUNDERSCOREbyUNDERSCOREsideUNDERSCOREfigure*}
[label = {evaluation-example},type = {Isa_COL.side_by_side_figure}, args={label = {evaluation-example},type = {Isa_COL.side_by_side_figure}, Isa_COL.side_by_side_figure.anchor = {fig-term-evaluation-ex}, Isa_COL.side_by_side_figure.caption = {The Evaluation succeeds.}, Isa_COL.figure.relative_width = {48}, Isa_COL.figure.src = {figures/term-context-evaluation-example}, Isa_COL.side_by_side_figure.anchor2 = {fig-term-failed-evaluation-ex}, Isa_COL.side_by_side_figure.caption2 = {The Evaluation fails.}, Isa_COL.side_by_side_figure.relative_width2 = {48}, Isa_COL.side_by_side_figure.src2 = {figures/term-context-failed-evaluation-example}, Isa_COL.figure.spawn_columns = {True}}]Evaluation of Antiquotations in a Term-Context.
\end{isamarkupsideUNDERSCOREbyUNDERSCOREsideUNDERSCOREfigure*}\isamarkuptrue%

\begin{isamarkuptext}
Since term antiquotations are logically uninterpreted constants,
  it is possible to compare class instances logically. The assertion
  in the \csname isaDof.ref\endcsname[type={Isa_COL.figure}]     {term-context-equality-evaluation} fails:
  the class instances \isa{proof{\isadigit{1}}} and \isa{proof{\isadigit{2}}} are not equivalent
  because their attribute \isa{property} differs.
  When \isa{\isacommand{assert{\isacharasterisk}{\kern0pt}}} evaluates the term,
  the term antiquotations \isa{{\isacharat}{\kern0pt}{\isacharbraceleft}{\kern0pt}thm\ {\isacharprime}{\kern0pt}{\isacharprime}{\kern0pt}HOL{\isachardot}{\kern0pt}refl{\isacharprime}{\kern0pt}{\isacharprime}{\kern0pt}{\isacharbraceright}{\kern0pt}} and \isa{{\isacharat}{\kern0pt}{\isacharbraceleft}{\kern0pt}thm\ {\isacharprime}{\kern0pt}{\isacharprime}{\kern0pt}HOL{\isachardot}{\kern0pt}sym{\isacharprime}{\kern0pt}{\isacharprime}{\kern0pt}{\isacharbraceright}{\kern0pt}} are checked
  against the global context such that the strings \isa{{\isacharprime}{\kern0pt}{\isacharprime}{\kern0pt}HOL{\isachardot}{\kern0pt}refl{\isacharprime}{\kern0pt}{\isacharprime}{\kern0pt}} and \isa{{\isacharprime}{\kern0pt}{\isacharprime}{\kern0pt}HOL{\isachardot}{\kern0pt}sym{\isacharprime}{\kern0pt}{\isacharprime}{\kern0pt}}
  denote existing theorems.
\end{isamarkuptext}\isamarkuptrue%

\begin{isamarkupfigure*}
[label = {term-context-equality-evaluation},type = {Isa_COL.figure}, args={label = {term-context-equality-evaluation},type = {Isa_COL.figure}, Isa_COL.figure.relative_width = {80}, Isa_COL.figure.src = {figures/term-context-equality-evaluation-example}, Isa_COL.figure.spawn_columns = {True}}]Evaluation of the Equivalence of two Class Instances.
\end{isamarkupfigure*}\isamarkuptrue%

\begin{isamarkuptext}
The mechanism of term annotations is also used for the new concept of 
  invariant constraints which can be specified in common HOL syntax.
  They were introduced by the keyword \isa{\isakeyword{invariant}}
  in a class definition (recall \autoref{fig-ontology-example}).
  Following the constraints proposed in \cite{brucker.ea:isabelle-ontologies:2018}, 
  one can specify that any instance of a class \isa{myresult}
  finally has a non-empty property list, if its \isa{kind} is \isa{proof}
  (see the \isa{\isakeyword{invariant}\ has{\isacharunderscore}{\kern0pt}property}), or that 
  the relation between \isa{myclaim} and \isa{myresult} expressed in the attribute \isa{establish}
  must be defined when an instance
  of the class \isa{myconclusion} is defined (see the \isa{\isakeyword{invariant}\ establish{\isacharunderscore}{\kern0pt}defined}).

  In \autoref{fig-ontology-example}, the \isa{\isakeyword{invariant}\ author{\isacharunderscore}{\kern0pt}set} of the class \isa{myintro} 
  enforces that a \isa{myintro} instance has at least one author.
  The \isa{{\isasymsigma}} symbol is reserved and references the future class instance.
  By relying on the implementation of the Records
  in Isabelle/HOL~\cite{wenzel:isabelle-isar:2020},
  one can reference an attribute of an instance using its selector function.
  For example, \isa{establish\ {\isasymsigma}} denotes the value
  of the attribute \isa{establish}
  of the future instance of the class \isa{myconclusion}.
\end{isamarkuptext}\isamarkuptrue%

\begin{isamarkuptext}
The value of each attribute defined for the instances is checked at run-time
  against their class invariants.
  Classes also inherit the invariants from their super-class.
  As the class \isa{myclaim} is a subclass
  of the class \isa{myintro}, it inherits the \isa{myintro} invariants.
  In \csname isaDof.ref\endcsname[type={Isa_COL.figure}]     {inherited-invariant-checking-figure},
  we attempt to specify a new instance \isa{claimNotion} of the class \isa{myclaim}.
  However, the invariant checking triggers an error because
  the \isa{\isakeyword{invariant}\ force{\isacharunderscore}{\kern0pt}level} forces the value of the argument
  of the attribute \isa{mytext{\isacharunderscore}{\kern0pt}section{\isachardot}{\kern0pt}level} to be greater than 1,
  and we initialize it to \isa{Some\ {\isadigit{0}}} in \isa{claimNotion}.
\end{isamarkuptext}\isamarkuptrue%

\begin{isamarkupfigure*}
[label = {inherited-invariant-checking-figure},type = {Isa_COL.figure}, args={label = {inherited-invariant-checking-figure},type = {Isa_COL.figure}, Isa_COL.figure.relative_width = {99}, Isa_COL.figure.src = {figures/inherited-invariant-checking-violated-example}, Isa_COL.figure.spawn_columns = {True}}]Inherited Invariant Violation.
\end{isamarkupfigure*}\isamarkuptrue%

\begin{isamarkuptext}
Any class definition generates term antiquotations checking a class instance reference
  in a particular logical context; these references were
  elaborated to objects they refer to.
  This paves the way for a new mechanism to query the ``current'' instances presented
  as a HOL \isa{list}.
  Arbitrarily complex queries can therefore be defined inside the logical language.
  Thus, to get the list of the properties of the instances of the class \isa{myresult},
  or to get the list of the authors of the instances of the  \isa{myintro} class,
  it suffices to treat this meta-data as usual:
  
\begin{isabelle}
\ \ \ \ \ \isacommand{value{\isacharasterisk}{\kern0pt}}{\isacartoucheopen}map\ {\isacharparenleft}{\kern0pt}myresult{\isachardot}{\kern0pt}property{\isacharparenright}{\kern0pt}\ {\isacharat}{\kern0pt}{\isacharbraceleft}{\kern0pt}myresult{\isacharminus}{\kern0pt}instances{\isacharbraceright}{\kern0pt}{\isacartoucheclose}\isanewline
\ \ \ \ \ \isacommand{value{\isacharasterisk}{\kern0pt}}{\isacartoucheopen}map\ {\isacharparenleft}{\kern0pt}mytext{\isacharunderscore}{\kern0pt}section{\isachardot}{\kern0pt}authored{\isacharunderscore}{\kern0pt}by{\isacharparenright}{\kern0pt}\ {\isacharat}{\kern0pt}{\isacharbraceleft}{\kern0pt}myintro{\isacharminus}{\kern0pt}instances{\isacharbraceright}{\kern0pt}{\isacartoucheclose}
\end{isabelle}
  In order to get the list of the instances of the class \isa{myresult}
  whose \isa{evidence} is a \isa{proof}, one can use the command:
  
\begin{isabelle}
\ \ \ \ \ \isacommand{value{\isacharasterisk}{\kern0pt}}{\isacartoucheopen}filter\ {\isacharparenleft}{\kern0pt}{\isasymlambda}{\isasymsigma}{\isachardot}{\kern0pt}\ myresult{\isachardot}{\kern0pt}evidence\ {\isasymsigma}\ {\isacharequal}{\kern0pt}\ proof{\isacharparenright}{\kern0pt}\ {\isacharat}{\kern0pt}{\isacharbraceleft}{\kern0pt}myresult{\isacharminus}{\kern0pt}instances{\isacharbraceright}{\kern0pt}{\isacartoucheclose}
\end{isabelle}
\end{isamarkuptext}\isamarkuptrue%

\begin{isamarkupsection*}
[label = {morphisms},type = {scholarly_paper.technical}, args={label = {morphisms},type = {scholarly_paper.technical}, scholarly_paper.text_section.main_author = {@{docitem ''idir''}}, Isa_COL.text_element.level = {}, Isa_COL.text_element.referentiable = {False}, Isa_COL.text_element.variants = {{STR ''outline'', STR ''document''}}, scholarly_paper.text_section.fixme_list = {}, Isa_COL.text_element.level = {}, scholarly_paper.technical.definition_list = {}, scholarly_paper.technical.status = {description}}]Proving Morphisms on Ontologies
\end{isamarkupsection*}\isamarkuptrue%
%
\isadelimproof
%
\endisadelimproof
%
\isatagproof
%
\endisatagproof
{\isafoldproof}%
%
\isadelimproof
%
\endisadelimproof
%
\isadelimproof
%
\endisadelimproof
%
\isatagproof
%
\endisatagproof
{\isafoldproof}%
%
\isadelimproof
%
\endisadelimproof

\begin{isamarkuptext}
The Isabelle/DOF framework does not assume that all documents refer to the same ontology. 
Each document may even build its local ontology without any external reference. 
It may also be based on several reference ontologies (\eg, from the  Isabelle/DOF library). 
Since ontological instances possess \emph{representations inside the logic},
the relationship between a local ontology and a reference ontology can be formalised
using a morphism function also inside the logic. More precisely, the instances of local ontology 
classes may be described as the image of a 
transformation applied to one or several other instances of class(es) belonging to another 
ontology. Thanks to the morphism relationship, the obtained class may either import meta-data 
(definitions are preserved)  or map meta-data (the properties are different but 
are semantically equivalent) that are defined in the referenced class(es). 
It may also provide additional properties. This means that morphisms may be injective, 
surjective, bijective, and thus describe abstract relations between ontologies.
This raises the question of invariant preservation.
\end{isamarkuptext}\isamarkuptrue%

\begin{isamarkuptext}
To illustrate this process, we have defined a simple ontology to classify Hardware objects.
%This ontology is described in \autoref{fig-Local-Ontology-example

%\begin{figure}[!h]

\begin{isarbox}

\begin{isabelle}
\isacommand{onto{\isacharunderscore}{\kern0pt}class}\ Item\ {\isacharequal}{\kern0pt}\isanewline
\ \ name\ {\isacharcolon}{\kern0pt}{\isacharcolon}{\kern0pt}\ string\isanewline
\isacommand{onto{\isacharunderscore}{\kern0pt}class}\ Product\ {\isacharequal}{\kern0pt}\ Item\ {\isacharplus}{\kern0pt}\isanewline
\ \ serial{\isacharunderscore}{\kern0pt}number\ {\isacharcolon}{\kern0pt}{\isacharcolon}{\kern0pt}\ int\isanewline
\ \ provider\ {\isacharcolon}{\kern0pt}{\isacharcolon}{\kern0pt}\ string\isanewline
\ \ mass\ {\isacharcolon}{\kern0pt}{\isacharcolon}{\kern0pt}\ int\isanewline
\isacommand{onto{\isacharunderscore}{\kern0pt}class}\ Computer{\isacharunderscore}{\kern0pt}Hardware\ {\isacharequal}{\kern0pt}\ Product\ {\isacharplus}{\kern0pt}\isanewline
\ \ type\ {\isacharcolon}{\kern0pt}{\isacharcolon}{\kern0pt}\ Hardware{\isacharunderscore}{\kern0pt}Type\isanewline
\ \ composed{\isacharunderscore}{\kern0pt}of\ {\isacharcolon}{\kern0pt}{\isacharcolon}{\kern0pt}\ {\isachardoublequoteopen}Product\ list{\isachardoublequoteclose}\ \isanewline
\ \ \isakeyword{invariant}\ c{\isadigit{2}}\ {\isacharcolon}{\kern0pt}{\isacharcolon}{\kern0pt}\ {\isachardoublequoteopen}Product{\isachardot}{\kern0pt}mass\ {\isasymsigma}\ {\isacharequal}{\kern0pt}\ sum{\isacharparenleft}{\kern0pt}map\ Product{\isachardot}{\kern0pt}mass\ {\isacharparenleft}{\kern0pt}composed{\isacharunderscore}{\kern0pt}of\ {\isasymsigma}{\isacharparenright}{\kern0pt}{\isacharparenright}{\kern0pt}{\isachardoublequoteclose}
\end{isabelle}
\end{isarbox}
%\caption{An extract of a local (user)  ontology.}
%\label{fig-Local-Ontology-example}
%\end{figure}
%} 

This ontology defines the \isa{Item}, \isa{Product} and \isa{Computer{\isacharunderscore}{\kern0pt}Hardware} concepts (or classes).  
Each class contains a set of attributes or properties and some local invariants.
In this example, we focus on the \isa{Computer{\isacharunderscore}{\kern0pt}Hardware} 
class defined as a list of products characterised by their mass value.
This class contains a local \isa{\isakeyword{invariant}\ c{\isadigit{2}}} to guarantee that its own mass value
equals the sum of all the masses of the products composing the object.
For the sake of the argument, we use the reference ontology (considered as a standard)
described in this listing:
%\begin{figure}

\begin{isarbox}

\begin{isabelle}
\isacommand{definition}\ sum\ \isakeyword{where}\ {\isachardoublequoteopen}sum\ S\ {\isacharequal}{\kern0pt}\ {\isacharparenleft}{\kern0pt}fold\ {\isacharparenleft}{\kern0pt}{\isacharplus}{\kern0pt}{\isacharparenright}{\kern0pt}\ S\ {\isadigit{0}}{\isacharparenright}{\kern0pt}{\isachardoublequoteclose}\isanewline
\isanewline
\isacommand{datatype}\ Hardware{\isacharunderscore}{\kern0pt}Type\ {\isacharequal}{\kern0pt}\ Motherboard\ {\isacharbar}{\kern0pt}\ Expansion{\isacharunderscore}{\kern0pt}Card\ {\isacharbar}{\kern0pt}\ Storage{\isacharunderscore}{\kern0pt}Device\ \isacommand{{\isachardot}{\kern0pt}{\isachardot}{\kern0pt}}\isacommand{{\isachardot}{\kern0pt}}\isanewline
\isanewline
\isacommand{onto{\isacharunderscore}{\kern0pt}class}\ Resource\ {\isacharequal}{\kern0pt}\isanewline
\ \ name\ {\isacharcolon}{\kern0pt}{\isacharcolon}{\kern0pt}\ string\isanewline
\isacommand{onto{\isacharunderscore}{\kern0pt}class}\ Electronic\ {\isacharequal}{\kern0pt}\ Resource\ {\isacharplus}{\kern0pt}\isanewline
\ \ provider\ {\isacharcolon}{\kern0pt}{\isacharcolon}{\kern0pt}\ string\isanewline
\ \ manufacturer\ {\isacharcolon}{\kern0pt}{\isacharcolon}{\kern0pt}\ string\isanewline
\isacommand{onto{\isacharunderscore}{\kern0pt}class}\ Component\ {\isacharequal}{\kern0pt}\ Electronic\ {\isacharplus}{\kern0pt}\isanewline
\ \ mass\ {\isacharcolon}{\kern0pt}{\isacharcolon}{\kern0pt}\ int\isanewline
\ \ dimensions\ {\isacharcolon}{\kern0pt}{\isacharcolon}{\kern0pt}\ {\isachardoublequoteopen}int\ list{\isachardoublequoteclose}\ \isanewline
\isacommand{onto{\isacharunderscore}{\kern0pt}class}\ Informatic\ {\isacharequal}{\kern0pt}\ Resource\ {\isacharplus}{\kern0pt}\isanewline
\ \ description\ {\isacharcolon}{\kern0pt}{\isacharcolon}{\kern0pt}\ string\isanewline
\isacommand{onto{\isacharunderscore}{\kern0pt}class}\ Hardware\ {\isacharequal}{\kern0pt}\ Informatic\ {\isacharplus}{\kern0pt}\isanewline
\ \ type\ {\isacharcolon}{\kern0pt}{\isacharcolon}{\kern0pt}\ Hardware{\isacharunderscore}{\kern0pt}Type\isanewline
\ \ mass\ {\isacharcolon}{\kern0pt}{\isacharcolon}{\kern0pt}\ int\isanewline
\ \ composed{\isacharunderscore}{\kern0pt}of\ {\isacharcolon}{\kern0pt}{\isacharcolon}{\kern0pt}\ {\isachardoublequoteopen}Component\ list{\isachardoublequoteclose}\ \isanewline
\ \ \isakeyword{invariant}\ c{\isadigit{1}}\ {\isacharcolon}{\kern0pt}{\isacharcolon}{\kern0pt}\ {\isachardoublequoteopen}mass\ {\isasymsigma}\ {\isacharequal}{\kern0pt}\ sum{\isacharparenleft}{\kern0pt}map\ Component{\isachardot}{\kern0pt}mass\ {\isacharparenleft}{\kern0pt}composed{\isacharunderscore}{\kern0pt}of\ {\isasymsigma}{\isacharparenright}{\kern0pt}{\isacharparenright}{\kern0pt}{\isachardoublequoteclose}
\end{isabelle}
\end{isarbox}
%\caption{An extract of a reference ontology.}
%\label{fig-Reference-Ontology-example}
%\end{figure}

This ontology defines  the \isa{Resource},
\isa{Electronic}, \isa{Component}, \isa{Informatic} and \isa{Hardware} concepts.
In our example, we focus on the \isa{Hardware} class containing a \isa{Component{\isachardot}{\kern0pt}mass} attribute
inherited from the \isa{Component} class
and composed of a list of components with a \isa{Component{\isachardot}{\kern0pt}mass} attribute formalising
the mass value of each component.
The \isa{Hardware} class also contains a local \isa{\isakeyword{invariant}\ c{\isadigit{1}}}
to define a constraint linking the global mass of a \isa{Hardware} object
with the masses of its own components.
\end{isamarkuptext}\isamarkuptrue%

\begin{isamarkuptext}
To check the coherence of our local ontology, we define a relationship between the local ontology 
and the reference ontology using morphism functions (or mapping rules as in ATL framework~\cite{atl}
or EXPRESS-X language~\cite{BGPP95}). These rules are applied to define the relationship 
between one class of the local ontology to one or several other class(es) described in the reference 
ontology. In our case, we have define two morphisms, \isa{Product{\isacharunderscore}{\kern0pt}to{\isacharunderscore}{\kern0pt}Component{\isacharunderscore}{\kern0pt}morphism}
and \isa{Computer{\isacharunderscore}{\kern0pt}Hardware{\isacharunderscore}{\kern0pt}to{\isacharunderscore}{\kern0pt}Hardware{\isacharunderscore}{\kern0pt}morphism}, detailed in the following listing: 

%\begin{figure}

\begin{isarbox}

\begin{isabelle}
\isacommand{definition}\ Product{\isacharunderscore}{\kern0pt}to{\isacharunderscore}{\kern0pt}Component{\isacharunderscore}{\kern0pt}morphism\ {\isacharcolon}{\kern0pt}{\isacharcolon}{\kern0pt}\ \isanewline
\ \ \ \ \ \ \ \ \ \ \ \ {\isachardoublequoteopen}Product\ {\isasymRightarrow}\ Component{\isachardoublequoteclose}\ {\isacharparenleft}{\kern0pt}{\isachardoublequoteopen}{\isacharunderscore}{\kern0pt}\ {\isasymlangle}Component{\isasymrangle}\isactrlsub P\isactrlsub r\isactrlsub o\isactrlsub d\isactrlsub u\isactrlsub c\isactrlsub t{\isachardoublequoteclose}\ {\isacharbrackleft}{\kern0pt}{\isadigit{1}}{\isadigit{0}}{\isadigit{0}}{\isadigit{0}}{\isacharbrackright}{\kern0pt}{\isadigit{9}}{\isadigit{9}}{\isadigit{9}}{\isacharparenright}{\kern0pt}\isanewline
\ \ \isakeyword{where}\ {\isachardoublequoteopen}\ \ {\isasymsigma}\ {\isasymlangle}Component{\isasymrangle}\isactrlsub P\isactrlsub r\isactrlsub o\isactrlsub d\isactrlsub u\isactrlsub c\isactrlsub t\ {\isacharequal}{\kern0pt}\ {\isasymlparr}\ Resource{\isachardot}{\kern0pt}tag{\isacharunderscore}{\kern0pt}attribute\ {\isacharequal}{\kern0pt}\ {\isadigit{1}}{\isacharcolon}{\kern0pt}{\isacharcolon}{\kern0pt}int\ {\isacharcomma}{\kern0pt}\isanewline
\ \ \ \ \ \ \ \ \ \ \ \ \ \ \ \ \ \ \ \ \ \ \ \ \ \ \ \ \ \ \ \ \ Resource{\isachardot}{\kern0pt}name\ {\isacharequal}{\kern0pt}\ name\ {\isasymsigma}\ {\isacharcomma}{\kern0pt}\isanewline
\ \ \ \ \ \ \ \ \ \ \ \ \ \ \ \ \ \ \ \ \ \ \ \ \ \ \ \ \ \ \ \ \ Electronic{\isachardot}{\kern0pt}provider\ \ {\isacharequal}{\kern0pt}\ provider\ {\isasymsigma}\ {\isacharcomma}{\kern0pt}\isanewline
\ \ \ \ \ \ \ \ \ \ \ \ \ \ \ \ \ \ \ \ \ \ \ \ \ \ \ \ \ \ \ \ \ Electronic{\isachardot}{\kern0pt}manufacturer\ \ {\isacharequal}{\kern0pt}\ {\isacharprime}{\kern0pt}{\isacharprime}{\kern0pt}no\ manufacturer{\isacharprime}{\kern0pt}{\isacharprime}{\kern0pt}\ {\isacharcomma}{\kern0pt}\isanewline
\ \ \ \ \ \ \ \ \ \ \ \ \ \ \ \ \ \ \ \ \ \ \ \ \ \ \ \ \ \ \ \ \ Component{\isachardot}{\kern0pt}mass\ {\isacharequal}{\kern0pt}\ mass\ {\isasymsigma}\ {\isacharcomma}{\kern0pt}\isanewline
\ \ \ \ \ \ \ \ \ \ \ \ \ \ \ \ \ \ \ \ \ \ \ \ \ \ \ \ \ \ \ \ \ Component{\isachardot}{\kern0pt}dimensions\ {\isacharequal}{\kern0pt}\ {\isacharbrackleft}{\kern0pt}{\isacharbrackright}{\kern0pt}\ {\isasymrparr}{\isachardoublequoteclose}
\end{isabelle}
\end{isarbox} 


\begin{isarbox}

\begin{isabelle}
\isacommand{definition}\ Computer{\isacharunderscore}{\kern0pt}Hardware{\isacharunderscore}{\kern0pt}to{\isacharunderscore}{\kern0pt}Hardware{\isacharunderscore}{\kern0pt}morphism\ {\isacharcolon}{\kern0pt}{\isacharcolon}{\kern0pt}\isanewline
\ \ \ \ \ \ \ \ \ \ \ \ {\isachardoublequoteopen}Computer{\isacharunderscore}{\kern0pt}Hardware\ {\isasymRightarrow}\ Hardware{\isachardoublequoteclose}\isanewline
\ \ \ \ \ \ \ \ \ \ \ {\isacharparenleft}{\kern0pt}{\isachardoublequoteopen}{\isacharunderscore}{\kern0pt}\ {\isasymlangle}Hardware{\isasymrangle}\isactrlsub C\isactrlsub o\isactrlsub m\isactrlsub p\isactrlsub u\isactrlsub t\isactrlsub e\isactrlsub r\isactrlsub H\isactrlsub a\isactrlsub r\isactrlsub d\isactrlsub w\isactrlsub a\isactrlsub r\isactrlsub e{\isachardoublequoteclose}\ {\isacharbrackleft}{\kern0pt}{\isadigit{1}}{\isadigit{0}}{\isadigit{0}}{\isadigit{0}}{\isacharbrackright}{\kern0pt}{\isadigit{9}}{\isadigit{9}}{\isadigit{9}}{\isacharparenright}{\kern0pt}\isanewline
\ \ \ \ \ \ \ \ \ \ \ \isakeyword{where}\ {\isachardoublequoteopen}{\isasymsigma}\ {\isasymlangle}Hardware{\isasymrangle}\isactrlsub C\isactrlsub o\isactrlsub m\isactrlsub p\isactrlsub u\isactrlsub t\isactrlsub e\isactrlsub r\isactrlsub H\isactrlsub a\isactrlsub r\isactrlsub d\isactrlsub w\isactrlsub a\isactrlsub r\isactrlsub e\ {\isacharequal}{\kern0pt}\isanewline
\ \ \ \ \ \ \ \ \ \ \ \ \ \ \ \ \ \ {\isasymlparr}\ \ Resource{\isachardot}{\kern0pt}tag{\isacharunderscore}{\kern0pt}attribute\ {\isacharequal}{\kern0pt}\ {\isadigit{2}}{\isacharcolon}{\kern0pt}{\isacharcolon}{\kern0pt}int\ {\isacharcomma}{\kern0pt}\isanewline
\ \ \ \ \ \ \ \ \ \ \ \ \ \ \ \ \ \ \ \ \ Resource{\isachardot}{\kern0pt}name\ {\isacharequal}{\kern0pt}\ name\ {\isasymsigma}\ {\isacharcomma}{\kern0pt}\isanewline
\ \ \ \ \ \ \ \ \ \ \ \ \ \ \ \ \ \ \ \ \ Informatic{\isachardot}{\kern0pt}description\ {\isacharequal}{\kern0pt}\ {\isacharprime}{\kern0pt}{\isacharprime}{\kern0pt}no\ description{\isacharprime}{\kern0pt}{\isacharprime}{\kern0pt}{\isacharcomma}{\kern0pt}\ \isanewline
\ \ \ \ \ \ \ \ \ \ \ \ \ \ \ \ \ \ \ \ \ Hardware{\isachardot}{\kern0pt}type\ {\isacharequal}{\kern0pt}\ type\ {\isasymsigma}\ {\isacharcomma}{\kern0pt}\isanewline
\ \ \ \ \ \ \ \ \ \ \ \ \ \ \ \ \ \ \ \ \ Hardware{\isachardot}{\kern0pt}mass\ {\isacharequal}{\kern0pt}\ mass\ {\isasymsigma}\ {\isacharcomma}{\kern0pt}\isanewline
\ \ \ \ \ \ \ \ \ \ \ \ \ \ \ \ \ \ \ \ \ Hardware{\isachardot}{\kern0pt}composed{\isacharunderscore}{\kern0pt}of\ {\isacharequal}{\kern0pt}\ map\ Product{\isacharunderscore}{\kern0pt}to{\isacharunderscore}{\kern0pt}Component{\isacharunderscore}{\kern0pt}morphism\ \isanewline
\ \ \ \ \ \ \ \ \ \ \ \ \ \ \ \ \ \ \ \ \ \ \ \ \ \ \ \ \ \ \ \ \ \ \ \ \ \ \ \ \ \ \ \ \ \ \ \ {\isacharparenleft}{\kern0pt}composed{\isacharunderscore}{\kern0pt}of\ {\isasymsigma}{\isacharparenright}{\kern0pt}\ {\isasymrparr}{\isachardoublequoteclose}
\end{isabelle}
\end{isarbox}
%\caption{An extract of a mapping definition.}
%\label{fig-mapping-example}
%\end{figure}


These definitions specify how \isa{Product} or \isa{Computer{\isacharunderscore}{\kern0pt}Hardware} objects are mapped to \isa{Component} 
or \isa{Hardware} objects defined in the reference ontology.
This mapping shows that the structure of a (user) ontology may be arbitrarily different
from the one of a standard ontology it references.
\end{isamarkuptext}\isamarkuptrue%

\begin{isamarkuptext}
After defining the mapping rules, now we have to deal with the question of invariant preservation. 
The following example proofs for a simple but typical example of reformatting meta-data into another 
format along an ontological mapping are nearly trivial:

%\begin{figure}

\begin{isarbox}

\begin{isabelle}
\isacommand{lemma}\ inv{\isacharunderscore}{\kern0pt}c{\isadigit{2}}{\isacharunderscore}{\kern0pt}preserved\ {\isacharcolon}{\kern0pt}\isanewline
\ \ {\isachardoublequoteopen}c{\isadigit{2}}{\isacharunderscore}{\kern0pt}inv\ {\isasymsigma}\ {\isasymLongrightarrow}\ c{\isadigit{1}}{\isacharunderscore}{\kern0pt}inv\ {\isacharparenleft}{\kern0pt}{\isasymsigma}\ {\isasymlangle}Hardware{\isasymrangle}\isactrlsub C\isactrlsub o\isactrlsub m\isactrlsub p\isactrlsub u\isactrlsub t\isactrlsub e\isactrlsub r\isactrlsub H\isactrlsub a\isactrlsub r\isactrlsub d\isactrlsub w\isactrlsub a\isactrlsub r\isactrlsub e{\isacharparenright}{\kern0pt}{\isachardoublequoteclose}\isanewline
\ \ \isacommand{unfolding}\ c{\isadigit{1}}{\isacharunderscore}{\kern0pt}inv{\isacharunderscore}{\kern0pt}def\ c{\isadigit{2}}{\isacharunderscore}{\kern0pt}inv{\isacharunderscore}{\kern0pt}def\ \isanewline
\ \ \ \ \ \ \ \ \ \ \ \ Computer{\isacharunderscore}{\kern0pt}Hardware{\isacharunderscore}{\kern0pt}to{\isacharunderscore}{\kern0pt}Hardware{\isacharunderscore}{\kern0pt}morphism{\isacharunderscore}{\kern0pt}def\isanewline
\ \ \ \ \ \ \ \ \ \ \ \ Product{\isacharunderscore}{\kern0pt}to{\isacharunderscore}{\kern0pt}Component{\isacharunderscore}{\kern0pt}morphism{\isacharunderscore}{\kern0pt}def\ \ \isanewline
\ \ \isacommand{by}\ {\isacharparenleft}{\kern0pt}auto\ simp{\isacharcolon}{\kern0pt}\ comp{\isacharunderscore}{\kern0pt}def{\isacharparenright}{\kern0pt}\isanewline
\isanewline
\isacommand{lemma}\ Computer{\isacharunderscore}{\kern0pt}Hardware{\isacharunderscore}{\kern0pt}to{\isacharunderscore}{\kern0pt}Hardware{\isacharunderscore}{\kern0pt}total\ {\isacharcolon}{\kern0pt}\isanewline
\ \ \ \ \ \ \ \ \ {\isachardoublequoteopen}Computer{\isacharunderscore}{\kern0pt}Hardware{\isacharunderscore}{\kern0pt}to{\isacharunderscore}{\kern0pt}Hardware{\isacharunderscore}{\kern0pt}morphism\ {\isacharbackquote}{\kern0pt}\ {\isacharparenleft}{\kern0pt}{\isacharbraceleft}{\kern0pt}X{\isachardot}{\kern0pt}\ c{\isadigit{2}}{\isacharunderscore}{\kern0pt}inv\ X{\isacharbraceright}{\kern0pt}{\isacharparenright}{\kern0pt}\ \isanewline
\ \ \ \ \ \ {\isasymsubseteq}\ {\isacharparenleft}{\kern0pt}{\isacharbraceleft}{\kern0pt}X{\isacharcolon}{\kern0pt}{\isacharcolon}{\kern0pt}Hardware{\isachardot}{\kern0pt}\ c{\isadigit{1}}{\isacharunderscore}{\kern0pt}inv\ X{\isacharbraceright}{\kern0pt}{\isacharparenright}{\kern0pt}{\isachardoublequoteclose}\isanewline
\ \ \isacommand{using}\ inv{\isacharunderscore}{\kern0pt}c{\isadigit{2}}{\isacharunderscore}{\kern0pt}preserved\ \isacommand{by}\ auto
\end{isabelle}
\end{isarbox}
%\caption{Proofs establishing an Invariant Preservation.}
%\label{fig-xxx}
%\end{figure}

After unfolding
the invariant and the morphism definitions, the preservation proof is automatic. The advantage 
of using the Isabelle/DOF framework compared to approaches like ATL or EXPRESS-X is 
the possibility of formally verifying the \emph{mapping rules}, \ie,  proving the preservation 
of invariants, as we have demonstrated in the previous example.
\end{isamarkuptext}\isamarkuptrue%

\begin{isamarkupsection*}
[label = {appl_certif},type = {scholarly_paper.technical}, args={label = {appl_certif},type = {scholarly_paper.technical}, Isa_COL.text_element.level = {}, Isa_COL.text_element.referentiable = {False}, Isa_COL.text_element.variants = {{STR ''outline'', STR ''document''}}, scholarly_paper.text_section.main_author = {}, scholarly_paper.text_section.fixme_list = {}, Isa_COL.text_element.level = {}, scholarly_paper.technical.definition_list = {}, scholarly_paper.technical.status = {status.description}}]Application: CENELEC Ontology
\end{isamarkupsection*}\isamarkuptrue%

\begin{isamarkuptext}
From its beginning, Isabelle/DOF had been used for documents containing formal models targeting 
certifications. A major case-study from the railways domain based on the CENELEC 50128 standard
had been published earlier (cf. \cite{DBLP:conf-ifm-BruckerW19}) 
\footnote{Our CENELEC ontology in 
Isabelle/DOF can be found at 
\url{https://github.com/logicalhacking/Isabelle_DOF/blob/main/src/ontologies/CENELEC_50128/CENELEC_50128.thy}.}.
The CENELEC Standard comprises 18 different ``Design and Test Documents''; a fully fledged description of 
our ontology covering these is therefore out of reach of this paper.
Rather, we present how the novel concepts such as invariants and term-antiquotations are used in 
selected elements in this ontology.

According to CENELEC Table C.1, for example, we specify the category of ``Design and Test Documents''
as follows:

\begin{isarbox}

\begin{isabelle}
\isacommand{doc{\isacharunderscore}{\kern0pt}class}\ cenelec{\isacharunderscore}{\kern0pt}document\ {\isacharequal}{\kern0pt}\ text{\isacharunderscore}{\kern0pt}element\ {\isacharplus}{\kern0pt}\isanewline
\ \ \ phase\ \ \ \ \ \ \ \ \ {\isacharcolon}{\kern0pt}{\isacharcolon}{\kern0pt}\ phase\isanewline
\ \ \ written{\isacharunderscore}{\kern0pt}by\ \ \ {\isacharcolon}{\kern0pt}{\isacharcolon}{\kern0pt}\ role\ \ \ \ \ \ %
\isamarkupcmt{Annex C Table C.1%
}\isanewline
\ \ \ fst{\isacharunderscore}{\kern0pt}check\ \ \ \ \ {\isacharcolon}{\kern0pt}{\isacharcolon}{\kern0pt}\ role\ \ \ \ \ \ %
\isamarkupcmt{Annex C Table C.1%
}\isanewline
\ \ \ snd{\isacharunderscore}{\kern0pt}check\ \ \ \ {\isacharcolon}{\kern0pt}{\isacharcolon}{\kern0pt}\ role\ \ \ \ \ \ %
\isamarkupcmt{Annex C Table C.1%
}\isanewline
\ \ \ \isacommand{{\isachardot}{\kern0pt}{\isachardot}{\kern0pt}}\isacommand{{\isachardot}{\kern0pt}}\ \isanewline
\ \ \ \isakeyword{invariant}\ must{\isacharunderscore}{\kern0pt}be{\isacharunderscore}{\kern0pt}chapter\ {\isacharcolon}{\kern0pt}{\isacharcolon}{\kern0pt}\ {\isachardoublequoteopen}text{\isacharunderscore}{\kern0pt}element{\isachardot}{\kern0pt}level\ {\isasymsigma}\ {\isacharequal}{\kern0pt}\ Some{\isacharparenleft}{\kern0pt}{\isadigit{0}}{\isacharparenright}{\kern0pt}{\isachardoublequoteclose}\isanewline
\ \ \ \isakeyword{invariant}\ two{\isacharunderscore}{\kern0pt}eyes{\isacharunderscore}{\kern0pt}prcple\ {\isacharcolon}{\kern0pt}{\isacharcolon}{\kern0pt}\ {\isachardoublequoteopen}written{\isacharunderscore}{\kern0pt}by\ {\isasymsigma}\ {\isasymnoteq}\ fst{\isacharunderscore}{\kern0pt}check\ {\isasymsigma}\ \isanewline
\ \ \ \ \ \ \ \ \ \ \ \ \ \ \ \ \ \ \ \ \ \ \ \ \ \ \ \ \ \ \ \ \ \ \ \ \ {\isasymand}\ written{\isacharunderscore}{\kern0pt}by\ {\isasymsigma}\ {\isasymnoteq}\ snd{\isacharunderscore}{\kern0pt}check\ {\isasymsigma}{\isachardoublequoteclose}
\end{isabelle}
\end{isarbox}

This class refers to the ``software phases'' the standard postulates (like \isa{SPl} for 
``Software Planning'' or \isa{SCDES} for ``Software Component Design'') 
which we model by a corresponding enumeration types (not shown here).  
Similarly, the standard postulates ``roles'' that certain specified teams possess in the overall process
(like \isa{VER} for verification and \isa{VAL} for validation). We added invariants that specify 
certain constraints implicit in the standard: for example, a \isa{cenelec{\isacharunderscore}{\kern0pt}document} must have
the textual structure of a chapter (the \emph{level}-attribute is inherited from an underlying 
ontology library specifying basic text-elements) as well as the two-eyes-principle between authors and
checkers of these chapters.
\end{isamarkuptext}\isamarkuptrue%

\begin{isamarkuptext}
The concrete sub-class of  \isa{cenelec{\isacharunderscore}{\kern0pt}document} is the class \isa{SWIS}
(``software interface specification'') as shown below, 
which provides the role assignment required for this document type: 

\begin{isarbox}

\begin{isabelle}
\isanewline
\isanewline
\isacommand{doc{\isacharunderscore}{\kern0pt}class}\ SWIS\ {\isacharequal}{\kern0pt}\ cenelec{\isacharunderscore}{\kern0pt}document\ {\isacharplus}{\kern0pt}\ %
\isamarkupcmt{software interface specification%
}\ \isanewline
\ \ \ phase\ \ \ \ \ \ \ \ {\isacharcolon}{\kern0pt}{\isacharcolon}{\kern0pt}\ {\isachardoublequoteopen}phase{\isachardoublequoteclose}\ \ {\isacharless}{\kern0pt}{\isacharequal}{\kern0pt}\ {\isachardoublequoteopen}SCDES{\isachardoublequoteclose}\ \ \ \ \ \ written{\isacharunderscore}{\kern0pt}by\ \ \ {\isacharcolon}{\kern0pt}{\isacharcolon}{\kern0pt}\ {\isachardoublequoteopen}role{\isachardoublequoteclose}\ \ \ {\isacharless}{\kern0pt}{\isacharequal}{\kern0pt}\ {\isachardoublequoteopen}DES{\isachardoublequoteclose}\isanewline
\ \ \ fst{\isacharunderscore}{\kern0pt}check\ \ \ \ {\isacharcolon}{\kern0pt}{\isacharcolon}{\kern0pt}\ {\isachardoublequoteopen}role{\isachardoublequoteclose}\ \ \ {\isacharless}{\kern0pt}{\isacharequal}{\kern0pt}\ {\isachardoublequoteopen}VER{\isachardoublequoteclose}\ \ \ \ \ \ \ \ \ \ \ snd{\isacharunderscore}{\kern0pt}check\ \ \ \ {\isacharcolon}{\kern0pt}{\isacharcolon}{\kern0pt}\ {\isachardoublequoteopen}role{\isachardoublequoteclose}\ \ \ {\isacharless}{\kern0pt}{\isacharequal}{\kern0pt}\ {\isachardoublequoteopen}VAL{\isachardoublequoteclose}\isanewline
\ \ \ components{\isacharcolon}{\kern0pt}{\isacharcolon}{\kern0pt}\ {\isachardoublequoteopen}SWIS{\isacharunderscore}{\kern0pt}E\ list{\isachardoublequoteclose}
\end{isabelle}
\end{isarbox} 
The structural constraints expressed so far can in principle be covered by any 
hand-coded validation process and suitable editing support (\eg,  Protégé \cite{protege}). 
However, a closer look at the class \isa{SWIS{\isacharunderscore}{\kern0pt}E} (``software interface specification 
element'') referenced in the \isa{components}-attribute reveals the unique power of Isabelle/DOF;
rather than saying ``there must be a pre-condition'', Isabelle/DOF can be far more precise:

\begin{isarbox}

\begin{isabelle}
\isacommand{doc{\isacharunderscore}{\kern0pt}class}\ SWIS{\isacharunderscore}{\kern0pt}E\ {\isacharequal}{\kern0pt}\isanewline
\ \ \ op{\isacharunderscore}{\kern0pt}name\ \ \ \ \ \ \ \ \ \ {\isacharcolon}{\kern0pt}{\isacharcolon}{\kern0pt}\ {\isachardoublequoteopen}string{\isachardoublequoteclose}\isanewline
\ \ \ op{\isacharunderscore}{\kern0pt}args{\isacharunderscore}{\kern0pt}res\ \ \ \ \ \ \ {\isacharcolon}{\kern0pt}{\isacharcolon}{\kern0pt}\ {\isachardoublequoteopen}{\isacharparenleft}{\kern0pt}string\ {\isasymtimes}\ typ{\isacharparenright}{\kern0pt}\ list\ {\isasymtimes}\ typ{\isachardoublequoteclose}\ %
\isamarkupcmt{args and result type%
}\isanewline
\ \ \ pre{\isacharunderscore}{\kern0pt}cond\ \ \ \ \ \ \ \ \ \ {\isacharcolon}{\kern0pt}{\isacharcolon}{\kern0pt}\ {\isachardoublequoteopen}{\isacharparenleft}{\kern0pt}string\ {\isasymtimes}\ thm{\isacharparenright}{\kern0pt}\ list{\isachardoublequoteclose}\ \ \ \ \ \ \ \ %
\isamarkupcmt{labels and predicates%
}\isanewline
\ \ \ post{\isacharunderscore}{\kern0pt}cond\ \ \ \ \ \ \ \ \ {\isacharcolon}{\kern0pt}{\isacharcolon}{\kern0pt}\ {\isachardoublequoteopen}{\isacharparenleft}{\kern0pt}string\ {\isasymtimes}\ thm{\isacharparenright}{\kern0pt}\ list{\isachardoublequoteclose}\ \ \ \ \ \ \ \ %
\isamarkupcmt{labels and predicates%
}\isanewline
\ \ \ \isakeyword{invariant}\ well{\isacharunderscore}{\kern0pt}formed{\isacharunderscore}{\kern0pt}pre\ {\isacharcolon}{\kern0pt}{\isacharcolon}{\kern0pt}\ {\isachardoublequoteopen}{\isasymforall}cond\ {\isasymin}\ set{\isacharparenleft}{\kern0pt}map\ snd\ {\isacharparenleft}{\kern0pt}pre{\isacharunderscore}{\kern0pt}cond\ {\isasymsigma}{\isacharparenright}{\kern0pt}{\isacharparenright}{\kern0pt}{\isachardot}{\kern0pt}\ \isanewline
\ \ \ \ \ \ \ \ \ \ \ \ \ \ \ \ \ \ \ \ \ \ \ \ \ \ \ \ \ \ \ \ \ \ \ \ \ \ \ \ \ \ \ \ \ \ iswff\isactrlsub p\isactrlsub r\isactrlsub e\ {\isacharparenleft}{\kern0pt}op{\isacharunderscore}{\kern0pt}args{\isacharunderscore}{\kern0pt}res\ {\isasymsigma}{\isacharparenright}{\kern0pt}\ {\isacharparenleft}{\kern0pt}cond{\isacharparenright}{\kern0pt}{\isachardoublequoteclose}\isanewline
\ \ \ \isakeyword{invariant}\ well{\isacharunderscore}{\kern0pt}formed{\isacharunderscore}{\kern0pt}post{\isacharcolon}{\kern0pt}{\isacharcolon}{\kern0pt}\ \isacommand{{\isachardot}{\kern0pt}{\isachardot}{\kern0pt}}\isacommand{{\isachardot}{\kern0pt}}
\end{isabelle}
\end{isarbox}
where the constant \isa{iswff\isactrlsub p\isactrlsub r\isactrlsub e} is bound to a function at the SML-level, that
is executed during the evaluation phase of these invariants and that checks:

%
\begin{itemize}%
\item Any \emph{cond} is indeed a valid definition in the global logical context
(taking HOL-libraries but also the concrete certification target model into account).

\item Any such HOL definition has the syntactic form:
\vspace{-0.3cm}

\begin{isabelle}
\ \ \ \ \ \ \ \ \ \ pre{\isacharunderscore}{\kern0pt}{\isacharless}{\kern0pt}op{\isacharunderscore}{\kern0pt}name{\isachargreater}{\kern0pt}\ {\isacharparenleft}{\kern0pt}a\isactrlsub {\isadigit{1}}{\isacharcolon}{\kern0pt}{\isacharcolon}{\kern0pt}{\isasymtau}\isactrlsub {\isadigit{1}}{\isacharparenright}{\kern0pt}\ {\isachardot}{\kern0pt}{\isachardot}{\kern0pt}{\isachardot}{\kern0pt}\ {\isacharparenleft}{\kern0pt}a\isactrlsub n{\isacharcolon}{\kern0pt}{\isacharcolon}{\kern0pt}{\isasymtau}\isactrlsub n{\isacharparenright}{\kern0pt}\ {\isasymequiv}\ {\isacharless}{\kern0pt}predicate{\isachargreater}{\kern0pt}{\isacharcomma}{\kern0pt}
\end{isabelle}
\vspace{-0.3cm}
where \isa{{\isacharparenleft}{\kern0pt}a\isactrlsub {\isadigit{1}}{\isacharcolon}{\kern0pt}{\isacharcolon}{\kern0pt}{\isasymtau}\isactrlsub {\isadigit{1}}{\isacharparenright}{\kern0pt}\ {\isachardot}{\kern0pt}{\isachardot}{\kern0pt}{\isachardot}{\kern0pt}\ {\isacharparenleft}{\kern0pt}a\isactrlsub n{\isacharcolon}{\kern0pt}{\isacharcolon}{\kern0pt}{\isasymtau}\isactrlsub n{\isacharparenright}{\kern0pt}} correspond to the argument list.

\item The case for the post-condition is treated analogously. %
\end{itemize}
\end{isamarkuptext}\isamarkuptrue%

\begin{isamarkuptext}
Note that this technique can also be applied to impose specific syntactic constraints on
types. For example, via the SI-package available in the Isabelle AFP
\footnote{\url{https://www.isa-afp.org/entries/Physical_Quantities.html}}, it is possible to express
that the result of some calculation is of type
\isa{{\isadigit{3}}{\isadigit{2}}\ unsigned\ {\isacharbrackleft}{\kern0pt}m\isactrlsup {\isasymcdot}s\isactrlsup {\isacharminus}{\kern0pt}\isactrlsup {\isadigit{2}}{\isacharbrackright}{\kern0pt}}, so a 32-bit natural representing an acceleration in the SI-system.
Therefore it is possible to impose that certain values refer to physical dimensions
measured in a concrete metrological system.
\end{isamarkuptext}\isamarkuptrue%

\begin{isamarkupsection*}
[label = {rw},type = {scholarly_paper.related_work}, args={label = {rw},type = {scholarly_paper.related_work}, Isa_COL.text_element.level = {}, Isa_COL.text_element.referentiable = {False}, Isa_COL.text_element.variants = {{STR ''outline'', STR ''document''}}, scholarly_paper.text_section.main_author = {}, scholarly_paper.text_section.fixme_list = {}, Isa_COL.text_element.level = {}, scholarly_paper.text_section.main_author = {}, scholarly_paper.text_section.main_author = {}}]Related Work
\end{isamarkupsection*}\isamarkuptrue%

\begin{isamarkuptext}
There are a number of approaches to use ontologies for structuring the link between
information and knowledge, and to make it amenable to 
 ``semantic'' search in or consistency checking of documents.
Some are targeting mathematical libraries,
like the search engine \url{http://shinh.org/wfs} which uses clever text-based search methods in 
a large number of formulas, agnostic of their logical context and of formal proof,
or the OAF project \cite{KohlhaseR21} which developed a common ontological format,
called  OMDoc/MMT, and six \emph{export} functions from major ITP systems into it.
The more difficult task to develop import functions has not been addressed, not to mention
the construction of imported proofs in a native tactic proof format. Rather, the emphasis
was put on building a server infrastructure based on conventional, rather heavy-weight
database and OWL technology. Our approach targets so far only one ITP system and
its libraries, and emphasizes type-safeness, expressive power and “depth” of meta-data,
which is adapted to the specific needs of ITP systems and theory developments.

There are also a number of proposals of ontologies targeting mathematics:
the OntoMath\textsuperscript{PRO} \cite{Nevzorova2014OntoMathPO} proposes a 
``taxonomy of the fields of mathematics'' (p. 6).
In total, OntoMath\textsuperscript{PRO} contains the daunting number of 3,449 classes,
which is in part due to the need to compensate
the lack of general datatype definition methods for meta-data.
It is nevertheless an interesting starting point for a future development of a mathematics ontology
in the Isabelle/DOF framework.
Other ontologies worth mentioning are DBpedia \cite{10.1007/978-3-540-76298-0_52}, 
which provides with the \emph{SPARQL endpoint}  \url{http://dbpedia.org/sparql} a search engine, 
and the more general ScienceWISE \footnote{\url{http://sciencewise.info/ontology/}.}
that allows users to introduce their own category concepts.
Both suffer from the lack of deeper meta-data modeling, and the latter is still at the beginning
(ScienceWISE marks the Mathematics part as ``under construction'').

Regarding the use of formal methods to formalise standards, the Event-B method was proposed by
Fotso et al. \cite{FotsoFLM18} for specifications of the hybrid ERTMS/ETCS level 3 standard,
in which requirements are specified using SysML/KAOS goal diagrams. The latter were translated into 
Event-B, where domain-specific properties were specified by ontologies.
In another case, Mendil et al. \cite{MendilASMP21} propose an Event-B framework for formalising standard
 conformance through formal modelling of standards as ontologies. 
The proposed approach was exemplified on ARINC 661 standard
in the context of a weather radar system application.
These works are essentially interested in expressing ontological concepts in a formal method 
but do not explicitly deal with the formalisation of rules/invariants defined in ontologies. 
The question of ontology-mappings is not addressed.

%   \url{https://github.com/CLLKazan/OntoMathPro} 
%
% ITEM The "Ontology for Engineering Mathematics" 
%  \url{https://tomgruber.org/writing/an-ontology-for-engineering-mathematics} is
%  is  unfortunately only a half-baked approach to model physical quantities
%  and SI-measurements. Instead of using ontologies for this purpose, there
%  exist approaches based on strong type systems
\end{isamarkuptext}\isamarkuptrue%

\begin{isamarkupsection*}
[label = {concl},type = {scholarly_paper.conclusion}, args={label = {concl},type = {scholarly_paper.conclusion}, Isa_COL.text_element.level = {}, Isa_COL.text_element.referentiable = {False}, Isa_COL.text_element.variants = {{STR ''outline'', STR ''document''}}, scholarly_paper.text_section.main_author = {}, scholarly_paper.text_section.fixme_list = {}, Isa_COL.text_element.level = {}, scholarly_paper.text_section.main_author = {}}]Conclusion and Future Work
\end{isamarkupsection*}\isamarkuptrue%

\begin{isamarkuptext}
We presented Isabelle/DOF, an ontology framework 
deeply integrating continuous-check\slash continuous-build functionality into
the formal development process in HOL. The novel feature of term-contexts in Isabelle/DOF,
which permits term-antiquotations elaborated in the parsing process, paves the
way for the abstract specification of meta-data constraints as well the possibility
of advanced search in the meta-data of document elements. Thus it profits and
extends  Isabelle's document-centric view on formal development.

Many ontological languages such as OWL as well as the meta-modeling technology 
available for UML/OCL provide concepts for semantic rules and constraints, but
leave the validation checking usually to external tools (if implementing them at all).
This limits their practical usefulness drastically. Our approach treats invariants as 
first-class citizens, and turns them into an object of formal study in, for example, 
ontological mappings. Such a technology exists, to our knowledge, for the first time.

Our experiments with adaptations of existing ontologies from engineering and mathematics
show that Isabelle/DOF's ODL has sufficient expressive power to cover all aspects
of languages such as OWL (with perhaps the exception of multiple inheritance on classes).
However, these ontologies have been developed specifically \emph{in} OWL and target
its specific support, the Protégé editor \cite{protege}. We argue that Isabelle/DOF might ask 
for a re-engineering of these ontologies: less deep hierarchies, rather deeper structure 
in meta-data and stronger invariants.
\end{isamarkuptext}\isamarkuptrue%

\begin{isamarkuptext}
We plan to complement Isabelle/DOF with incremental LaTeX generation and a previewing facility
that will further increase the usability of our framework for the ontology-conform editing
of formal content, be it in the engineering or the mathematics domain
(this paper has been edited in Isabelle/DOF, of course).

Another line of future application is to increase the ``depth'' of built-in term antiquotations such
as \isa{{\isacharat}{\kern0pt}\isacommand{{\isacharbraceleft}{\kern0pt}}\isacommand{typ}\ {\isacartoucheopen}{\isacharprime}{\kern0pt}{\isasymtau}{\isacartoucheclose}\isacommand{{\isacharbraceright}{\kern0pt}}}, \isa{{\isacharat}{\kern0pt}\isacommand{{\isacharbraceleft}{\kern0pt}}\isacommand{term}\ {\isacartoucheopen}a\ {\isacharplus}{\kern0pt}\ b{\isacartoucheclose}\isacommand{{\isacharbraceright}{\kern0pt}}} and \isa{{\isacharat}{\kern0pt}\isacommand{{\isacharbraceleft}{\kern0pt}}\isacommand{thm}\ {\isacartoucheopen}HOL{\isachardot}{\kern0pt}refl{\isacartoucheclose}\isacommand{{\isacharbraceright}{\kern0pt}}}, which are currently implemented 
just as validations in the current logical context. In the future, they could  optionally be expanded 
to the types, terms and theorems (with proof objects attached) in a meta-model of the Isabelle Kernel 
such as the one presented in \cite{10.1007/978-3-030-79876-5_6} (also available in the AFP). 
This will allow for definitions of query-functions in, \eg, proof-objects, and pave the way 
to annotate them with typed meta-data. Such a technology could be relevant for the interoperability 
of proofs across different ITP platforms.
\end{isamarkuptext}\isamarkuptrue%
%
\isadelimtheory
%
\endisadelimtheory
%
\isatagtheory
%
\endisatagtheory
{\isafoldtheory}%
%
\isadelimtheory
%
\endisadelimtheory
%
\end{isabellebody}%
\endinput
%:%file=paper.tex%:%
%:%32=81%:%
%:%36=86%:%
%:%40=91%:%
%:%44=94%:%
%:%48=98%:%
%:%49=99%:%
%:%50=100%:%
%:%51=101%:%
%:%52=102%:%
%:%53=103%:%
%:%54=104%:%
%:%55=105%:%
%:%56=106%:%
%:%57=107%:%
%:%58=108%:%
%:%59=109%:%
%:%60=110%:%
%:%61=111%:%
%:%62=112%:%
%:%63=113%:%
%:%64=114%:%
%:%68=116%:%
%:%72=118%:%
%:%73=119%:%
%:%74=120%:%
%:%75=121%:%
%:%76=122%:%
%:%77=123%:%
%:%78=124%:%
%:%79=125%:%
%:%80=126%:%
%:%81=127%:%
%:%82=128%:%
%:%83=129%:%
%:%84=130%:%
%:%85=131%:%
%:%88=134%:%
%:%89=135%:%
%:%90=136%:%
%:%91=137%:%
%:%92=138%:%
%:%93=139%:%
%:%94=140%:%
%:%95=141%:%
%:%96=142%:%
%:%97=143%:%
%:%98=144%:%
%:%99=145%:%
%:%100=146%:%
%:%101=147%:%
%:%105=149%:%
%:%107=150%:%
%:%108=151%:%
%:%110=152%:%
%:%111=153%:%
%:%113=154%:%
%:%115=155%:%
%:%116=156%:%
%:%117=157%:%
%:%118=158%:%
%:%123=161%:%
%:%124=162%:%
%:%125=163%:%
%:%126=164%:%
%:%127=165%:%
%:%128=166%:%
%:%129=167%:%
%:%130=168%:%
%:%131=169%:%
%:%132=170%:%
%:%133=171%:%
%:%134=172%:%
%:%135=173%:%
%:%141=177%:%
%:%142=178%:%
%:%143=179%:%
%:%144=180%:%
%:%145=181%:%
%:%146=182%:%
%:%147=183%:%
%:%148=184%:%
%:%149=185%:%
%:%150=186%:%
%:%151=187%:%
%:%152=188%:%
%:%153=189%:%
%:%154=190%:%
%:%159=193%:%
%:%160=194%:%
%:%165=197%:%
%:%166=198%:%
%:%167=199%:%
%:%168=200%:%
%:%169=201%:%
%:%170=202%:%
%:%171=203%:%
%:%172=204%:%
%:%176=234%:%
%:%180=235%:%
%:%181=236%:%
%:%182=237%:%
%:%184=238%:%
%:%185=239%:%
%:%186=240%:%
%:%191=242%:%
%:%196=245%:%
%:%197=246%:%
%:%198=247%:%
%:%199=248%:%
%:%200=249%:%
%:%201=250%:%
%:%202=251%:%
%:%203=252%:%
%:%204=253%:%
%:%205=254%:%
%:%206=255%:%
%:%207=256%:%
%:%208=257%:%
%:%209=258%:%
%:%210=259%:%
%:%211=260%:%
%:%215=266%:%
%:%219=286%:%
%:%223=288%:%
%:%223=289%:%
%:%224=290%:%
%:%225=291%:%
%:%226=292%:%
%:%227=293%:%
%:%228=294%:%
%:%229=295%:%
%:%230=296%:%
%:%231=297%:%
%:%232=298%:%
%:%233=299%:%
%:%234=300%:%
%:%235=301%:%
%:%236=302%:%
%:%237=303%:%
%:%238=304%:%
%:%239=305%:%
%:%243=322%:%
%:%247=324%:%
%:%248=325%:%
%:%252=326%:%
%:%254=327%:%
%:%255=328%:%
%:%257=329%:%
%:%258=330%:%
%:%259=331%:%
%:%260=332%:%
%:%265=334%:%
%:%266=335%:%
%:%267=336%:%
%:%268=337%:%
%:%269=338%:%
%:%270=339%:%
%:%271=340%:%
%:%272=341%:%
%:%277=344%:%
%:%278=345%:%
%:%279=346%:%
%:%280=347%:%
%:%284=353%:%
%:%288=355%:%
%:%289=356%:%
%:%290=357%:%
%:%291=358%:%
%:%292=359%:%
%:%296=396%:%
%:%297=397%:%
%:%301=398%:%
%:%303=399%:%
%:%308=401%:%
%:%309=402%:%
%:%310=403%:%
%:%311=404%:%
%:%312=405%:%
%:%321=407%:%
%:%333=408%:%
%:%334=409%:%
%:%338=410%:%
%:%340=411%:%
%:%345=413%:%
%:%346=414%:%
%:%347=415%:%
%:%348=416%:%
%:%349=417%:%
%:%350=418%:%
%:%351=419%:%
%:%355=444%:%
%:%359=503%:%
%:%360=504%:%
%:%361=505%:%
%:%362=506%:%
%:%363=507%:%
%:%364=508%:%
%:%365=509%:%
%:%366=510%:%
%:%367=511%:%
%:%368=512%:%
%:%372=513%:%
%:%374=514%:%
%:%376=515%:%
%:%377=516%:%
%:%379=517%:%
%:%380=518%:%
%:%381=519%:%
%:%383=520%:%
%:%385=521%:%
%:%389=522%:%
%:%391=523%:%
%:%392=524%:%
%:%396=526%:%
%:%397=527%:%
%:%398=528%:%
%:%399=529%:%
%:%400=530%:%
%:%401=531%:%
%:%402=532%:%
%:%403=533%:%
%:%407=541%:%
%:%408=542%:%
%:%409=543%:%
%:%439=568%:%
%:%440=569%:%
%:%441=570%:%
%:%442=571%:%
%:%443=572%:%
%:%444=573%:%
%:%445=574%:%
%:%458=582%:%
%:%459=583%:%
%:%460=584%:%
%:%461=585%:%
%:%462=586%:%
%:%463=587%:%
%:%464=588%:%
%:%465=589%:%
%:%466=590%:%
%:%467=591%:%
%:%468=592%:%
%:%469=593%:%
%:%470=594%:%
%:%474=608%:%
%:%478=615%:%
%:%479=616%:%
%:%480=617%:%
%:%481=618%:%
%:%482=619%:%
%:%486=632%:%
%:%490=657%:%
%:%491=658%:%
%:%492=659%:%
%:%493=660%:%
%:%494=661%:%
%:%495=662%:%
%:%496=663%:%
%:%497=664%:%
%:%498=665%:%
%:%502=672%:%
%:%506=676%:%
%:%507=677%:%
%:%508=678%:%
%:%509=679%:%
%:%510=680%:%
%:%511=681%:%
%:%512=682%:%
%:%513=683%:%
%:%514=684%:%
%:%515=685%:%
%:%516=686%:%
%:%517=687%:%
%:%518=688%:%
%:%519=689%:%
%:%520=690%:%
%:%521=691%:%
%:%522=692%:%
%:%523=693%:%
%:%524=694%:%
%:%525=695%:%
%:%526=696%:%
%:%530=704%:%
%:%531=705%:%
%:%532=706%:%
%:%533=707%:%
%:%534=708%:%
%:%535=709%:%
%:%536=710%:%
%:%537=711%:%
%:%538=712%:%
%:%539=713%:%
%:%540=714%:%
%:%544=721%:%
%:%548=733%:%
%:%549=734%:%
%:%550=735%:%
%:%551=736%:%
%:%552=737%:%
%:%553=738%:%
%:%554=739%:%
%:%555=740%:%
%:%556=741%:%
%:%557=742%:%
%:%561=744%:%
%:%562=745%:%
%:%563=746%:%
%:%564=747%:%
%:%571=751%:%
%:%601=840%:%
%:%602=841%:%
%:%603=842%:%
%:%604=843%:%
%:%605=844%:%
%:%606=845%:%
%:%607=846%:%
%:%608=847%:%
%:%609=848%:%
%:%610=849%:%
%:%611=850%:%
%:%612=851%:%
%:%613=852%:%
%:%614=853%:%
%:%618=856%:%
%:%619=857%:%
%:%620=858%:%
%:%621=859%:%
%:%637=870%:%
%:%638=871%:%
%:%639=872%:%
%:%640=873%:%
%:%641=874%:%
%:%642=875%:%
%:%643=876%:%
%:%644=877%:%
%:%645=878%:%
%:%646=879%:%
%:%647=880%:%
%:%648=881%:%
%:%649=882%:%
%:%650=883%:%
%:%651=884%:%
%:%676=904%:%
%:%677=905%:%
%:%678=906%:%
%:%679=907%:%
%:%680=908%:%
%:%681=909%:%
%:%682=910%:%
%:%683=911%:%
%:%684=912%:%
%:%685=913%:%
%:%686=914%:%
%:%687=915%:%
%:%688=916%:%
%:%689=917%:%
%:%693=922%:%
%:%694=923%:%
%:%695=924%:%
%:%696=925%:%
%:%697=926%:%
%:%698=927%:%
%:%699=928%:%
%:%700=929%:%
%:%714=938%:%
%:%715=939%:%
%:%732=951%:%
%:%733=952%:%
%:%734=953%:%
%:%735=954%:%
%:%736=955%:%
%:%737=956%:%
%:%738=957%:%
%:%739=958%:%
%:%740=959%:%
%:%741=960%:%
%:%745=964%:%
%:%746=965%:%
%:%747=966%:%
%:%748=967%:%
%:%749=968%:%
%:%766=980%:%
%:%767=981%:%
%:%768=982%:%
%:%769=983%:%
%:%770=984%:%
%:%771=985%:%
%:%772=986%:%
%:%773=987%:%
%:%774=988%:%
%:%775=989%:%
%:%779=1052%:%
%:%783=1054%:%
%:%784=1055%:%
%:%785=1056%:%
%:%786=1057%:%
%:%787=1058%:%
%:%788=1059%:%
%:%789=1060%:%
%:%790=1061%:%
%:%791=1062%:%
%:%792=1063%:%
%:%793=1064%:%
%:%794=1065%:%
%:%795=1066%:%
%:%816=1076%:%
%:%817=1077%:%
%:%818=1078%:%
%:%819=1079%:%
%:%820=1080%:%
%:%821=1081%:%
%:%822=1082%:%
%:%823=1083%:%
%:%824=1084%:%
%:%825=1085%:%
%:%826=1086%:%
%:%830=1088%:%
%:%831=1089%:%
%:%832=1090%:%
%:%846=1096%:%
%:%847=1097%:%
%:%848=1098%:%
%:%849=1099%:%
%:%850=1100%:%
%:%851=1101%:%
%:%872=1111%:%
%:%873=1112%:%
%:%877=1113%:%
%:%878=1114%:%
%:%880=1115%:%
%:%881=1116%:%
%:%886=1118%:%
%:%887=1119%:%
%:%889=1120%:%
%:%894=1122%:%
%:%895=1123%:%
%:%896=1124%:%
%:%897=1125%:%
%:%898=1126%:%
%:%899=1127%:%
%:%900=1128%:%
%:%904=1131%:%
%:%908=1133%:%
%:%909=1134%:%
%:%910=1135%:%
%:%911=1136%:%
%:%912=1137%:%
%:%913=1138%:%
%:%914=1139%:%
%:%915=1140%:%
%:%916=1141%:%
%:%917=1142%:%
%:%918=1143%:%
%:%919=1144%:%
%:%920=1145%:%
%:%921=1146%:%
%:%922=1147%:%
%:%923=1148%:%
%:%924=1149%:%
%:%925=1150%:%
%:%926=1151%:%
%:%927=1152%:%
%:%928=1153%:%
%:%929=1154%:%
%:%930=1155%:%
%:%931=1156%:%
%:%932=1157%:%
%:%933=1158%:%
%:%934=1159%:%
%:%935=1160%:%
%:%936=1161%:%
%:%937=1162%:%
%:%938=1163%:%
%:%939=1164%:%
%:%940=1165%:%
%:%941=1166%:%
%:%942=1167%:%
%:%943=1168%:%
%:%944=1169%:%
%:%945=1170%:%
%:%946=1171%:%
%:%947=1172%:%
%:%948=1173%:%
%:%949=1174%:%
%:%950=1175%:%
%:%951=1176%:%
%:%952=1177%:%
%:%953=1178%:%
%:%954=1179%:%
%:%955=1180%:%
%:%956=1181%:%
%:%960=1184%:%
%:%964=1185%:%
%:%965=1186%:%
%:%966=1187%:%
%:%967=1188%:%
%:%968=1189%:%
%:%969=1190%:%
%:%970=1191%:%
%:%971=1192%:%
%:%972=1193%:%
%:%973=1194%:%
%:%974=1195%:%
%:%975=1196%:%
%:%976=1197%:%
%:%977=1198%:%
%:%978=1199%:%
%:%979=1200%:%
%:%980=1201%:%
%:%981=1202%:%
%:%982=1203%:%
%:%983=1204%:%
%:%984=1205%:%
%:%985=1206%:%
%:%989=1211%:%
%:%990=1212%:%
%:%991=1213%:%
%:%992=1214%:%
%:%993=1215%:%
%:%994=1216%:%
%:%995=1217%:%
%:%996=1218%:%
%:%997=1219%:%
%:%998=1220%:%
%:%999=1221%:%
%:%1000=1222%:%
%:%1001=1223%:%



% optional bibliography
\bibliographystyle{abbrv}
\bibliography{adb-long,root}

\end{document}

%%% Local Variables:
%%% mode: latex
%%% TeX-master: t
%%% End:
