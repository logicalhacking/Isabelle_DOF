\documentclass[11pt,a4paper]{book}
\usepackage{isabelle,isabellesym}
\usepackage{graphicx}
\graphicspath {{figures/}}

% further packages required for unusual symbols (see also
% isabellesym.sty), use only when needed


\usepackage{latexsym}
\usepackage{amssymb}
  %for \<leadsto>, \<box>, \<diamond>, \<sqsupset>, \<mho>, \<Join>,
  %\<lhd>, \<lesssim>, \<greatersim>, \<lessapprox>, \<greaterapprox>,
  %\<triangleq>, \<yen>, \<lozenge>

%\usepackage[greek,english]{babel}
  %option greek for \<euro>
  %option english (default language) for \<guillemotleft>, \<guillemotright>

%\usepackage[latin1]{inputenc}
  %for \<onesuperior>, \<onequarter>, \<twosuperior>, \<onehalf>,
  %\<threesuperior>, \<threequarters>, \<degree>

%\usepackage[only,bigsqcap]{stmaryrd}
  %for \<Sqinter>

%\usepackage{eufrak}
  %for \<AA> ... \<ZZ>, \<aa> ... \<zz> (also included in amssymb)

%\usepackage{textcomp}
  %for \<cent>, \<currency>

% this should be the last package used
\usepackage{pdfsetup}

% urls in roman style, theory text in math-similar italics
\urlstyle{rm}
\isabellestyle{it}
\newcommand{\HOL}[1]{\verb{HOL}}
\newcommand{\eg}[1]{e.g.}
\renewcommand{\isasymdegree}{XXX}


\begin{document}

\title{Type and Proof Support for SI Units}
\author{ Simon Foster \and Burkhart Wolff}
\maketitle

\textbf{ Abstract } 
The International System of Units 
(SI, abbreviated from the French Syst\`eme international (d’unit\'es)) is the modern form of 
the metric system and is the most widely used system of measurement. It comprises a coherent 
system of units of measurement built on seven base units, which are the second, metre, kilogram, 
ampere, kelvin, mole, candela, and a set of twenty prefixes to the unit names and unit symbols 
that may be used when specifying multiples and fractions of the units. The system also specifies 
names for 22 derived units, such as lumen and watt, for other common physical quantities. 

This theory represents a formal model of SI Units together with a deep integration in Isabelle's
type system: unitswere represented in a way that they have a \emph{unit type} comprising its 
\emph{magnitude type} and its physical \emph{dimension}. Congruences on dimensions were supported. 
Our construction is validated by a test-set of known congruences between SI Units.

\tableofcontents

% sane default for proof documents
\parindent 0pt\parskip 0.5ex


\chapter{}

The International System of Units (SI, abbreviated from the French
Système international (d'unités)) is the modern form of the metric
system and is the most widely used system of measurement. It comprises
a coherent system of units of measurement built on seven base units,
which are the second, metre, kilogram, ampere, kelvin, mole, candela,
and a set of twenty prefixes to the unit names and unit symbols that
may be used when specifying multiples and fractions of the units.
The system also specifies names for 22 derived units, such as lumen and
watt, for other common physical quantities.

(cited from \url{https://en.wikipedia.org/wiki/International_System_of_Units}).

In more detail, the SI provides the following fundamental concepts:

%
\begin{enumerate}%
\item \emph{quantities}, i.e. \emph{time}, \emph{length}, \emph{mass}, \emph{electric current},
\emph{temperature}, \emph{amount of substance},\emph{luminous intensity},
and other derived quantities such as \emph{volume};

\item \emph{dimensions}, i.e. a set of the symbols  \isa{T}, \isa{L}, \isa{M}, \isa{I},  \isa{{\isasymTheta}}, \isa{N}, \isa{J}  corresponding
to the above mentioned base quantities,  indexed by an integer exponent
(dimensions were also called \emph{base unit names} or just \emph{base units});

\item \emph{magnitudes}, i.e. a factor or \emph{prefix}
(typically integers, reals, vectors on real or complex numbers);

\item \emph{units}, which are basically pairs of magnitudes and dimensions denoting quantities.
\end{enumerate}


The purpose of this theory is to model SI units with polymorphic magnitudes in terms of the
Isabelle/HOL type system. The objective of this construction is reflecting the types of the
magnitudes as well as their dimensions in order to allow type-safe calculations on SI units.

Thus, it is possible to express "4500.0 kilogram times meter per second square" which will
have the type \isa{{\isasymreal}\ {\isacharbrackleft}M\ \isactrlsup {\isachardot}\ L\ \isactrlsup {\isachardot}\ T\isactrlsup {\isacharminus}\isactrlsup {\isadigit{2}}}, which can be used to infer that this corresponds to the derived
unit "4.5 kN" (kilo-Newton).  %
%
This is an attempt to model the system and its derived entities (cf.
\url{https://www.quora.com/What-are-examples-of-SI-units}) in Isabelle/HOL.
The design objective are twofold (and for the case of Isabelle somewhat
contradictory, see below).

The construction proceeds in three phases:

\begin{enumerate}%
\item We construct a type \isa{Dimension} which is basically a "semantic representation" or
"semantic domain" of all SI dimensions. Since SI-types have an interpretation in this domain,
it serves to give semantics to type-constructors by operations on this domain, too.
We construct a multiplicative group on it.

\item From \isa{Unit} we build a  \isa{{\isacharprime}a\ SI{\isacharunderscore}tagged{\isacharunderscore}domain} types, i.e. a polymorphic family of values
tagged with values from \isa{Unit}. We construct multiplicative and additive
groups over it.

\item We construct a type-class characterizing SI - type expressions
and types tagged with SI - type expressions; this construction paves the
way to overloaded interpretation functions from SI type-expressions to

\end{enumerate}%

\cite{nipkow.ea:isabelle:2002}


% generated text of all theories
\chapter{Appendix: The Theories}

\input{session}

% optional bibliography
\bibliographystyle{abbrv}
\bibliography{adb-long,root}

\end{document}

%%% Local Variables:
%%% mode: latex
%%% TeX-master: t
%%% End:
